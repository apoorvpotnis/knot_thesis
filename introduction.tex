\chapter{Introduction to knot theory}

\section{Definition of a knot}

We normally conceive of knots as open strings with `knotted' parts in between. Given a knotted string with two `open' ends, we can simply pull one end of the string to untie it, in the usual way we untie a knot. This way, any knot with two open ends can be untied. But if have a knot in a closed loop, we won't be able to untie it unless we cut it. We want our notion of a knot to be invariant under `pulling'. Thus, we model our mathematical definition on closed loops instead of open. Insert a figure illustrating this.
\begin{defn}
    A knot \(K\) is an embedding of \(\sone\) in \(\rthree\).
\end{defn}
Although one can talk about embeddings of higher dimensional `circles' in higher dimensional spaces, we restrict ourselves to knots in the three dimensional space. The above definition turns out out to be too general for our purposes. It takes into consideration certain pathological knots known as \textit{wild knots}. Although wild knots are an object of study, we won't be dealing with them in this thesis due to their pathological nature. fig shows an example of a wild knot. A section of the knot is scaled down by a constant factor joined on the one side of the previous section. If one does repeats this process infinitely, the section eventually converges to a point, provided that the scaling down is fast enough. We can join the convergence (limit) point to a an end-section on the other side to get wild knot. This knot is continuous everywhere, including the limit point, and an embedding of \(\sone\) in \(\rthree\). Three common ways exist to exclude such behaviour by demanding extra conditions.
\begin{enumerate}
    \item Differentiability.

    We can demand all knots to be differentiable at all points. We can verify visually that all the points, except the limit point of the wild knot are differentiable, or can be made (isotoped) differentiable. The derivative must necessarily change within a section. As the size of the section decreases, the derivative changes more rapidly. This rate of change of the derivative, the double derivative, is a monotonically increasing function as the section number increases. At the limit point, the derivative shall cease to exist by the virtue of `changing too rapidly'. Demanding differentiability forcibly removes the offending limit point. The `wildness' is due to the limit point.

    But this condition comes with a problem as well, namely we cannot use polygons for describing knots.
    \item Piecewise linearity.
    We can demand all the edges of a knot to be piecewise linear. Our knot shall be a polynomial in that case. A polygon has finitely many edges. Infinitely (countably) many sections in a wild knot shall mean infinitely many edges (of decreasing length), which is not allowed. Thus, this condition excludes wild knots.
    \item Local flatness.
\end{enumerate}
In this thesis, we shall take the third route following Cromwell~\cite[chp.~1]{cromwell}. If we consider a local neighbourhood around each point of the knot, except the limit point, then we see visually see that we can always find a small enough local neighbourhood around each such point such that the strand is not `knotted' in that neighbourhood. At the limit point, no matter how small a neighbourhood we take, the strand shall always be `knotted'. We enforce this `local unknottedness' condition by demanding local flatness. Let \(p\) be a point in a knot \(K\), \(\symup{B}(0, 1)\) be the unit ball centered at origin and \(d\) be a diameter of \(\symup{B}(0, 1)\).
\begin{defn}[Local flatness]
    The point \(p\) is said to be locally flat if there exists a neighbourhood \(U \ni p\) such that the pair \((U, U \cap K)\) is homeomorphic to \((\symup{B}, d)\).

    A knot is said to be locally flat if each point in that knot is locally flat. A point that is not locally flat is called wild, and a knot is wild if any of its points are wild.
\end{defn}
Insert figures of a unit ball centered at origin and a diameter, and a section of the knot homeomorphic to the pair.

Consider a spherical neighbourhood around a locally flat point. There exists a radius such that for all neighbourhoods less than this radius, the boundary of the neighbourhood intersects the strand in exactly two points. This is not possible at the limit point in wild knot figure.

\begin{defn}[Tame knots]
    A knot is said to be tame if all its points are locally flat.
\end{defn}
Unless mentioned otherwise, we shall always consider our knots to be tame from now on.



%
%
% \section{Isotopy}
%
% \subsection{Bachelors' unknotting}

% \begin{figure}
%     \centering
% 	\subcaptionbox{}{
% 		\begin{tikzpicture}[scale=0.5]
% 			\begin{knot}[clip width=5, consider self intersections,
% 				ignore endpoint intersections=false]
% 				\strand[thick, red] (7,3) to [curve through={(5,7) .. (2.7,5.2) .. (4,4) ..  (5.6,4.3) .. (6.9,5.5) .. (7,6) .. (6, 8).. (5,5)}] (7.5, 4);
% 				\flipcrossings{1,3}
% 			\end{knot}
%
% 		\end{tikzpicture}
%
% }
% \end{figure}
% .. (6.1,10.3) .. (4.4,10.3) .. (3.2,9.6) .. (4.4,10.3)

