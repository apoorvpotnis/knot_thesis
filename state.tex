\chapter{The Bracket Polynomial}

Given two links or knots, one desires a way to tell if the two links are distinct. Since we can represent knots faithfully on a paper using knot diagrams, one can ask for a way to distinguish knots based on their diagrams. A \textit{link invariant} is a function of a link such that if the evaluation of the function on two links yields different outputs, then the links are distinct, i.e.\@ not ambient isotopic to each other. A link invariant is said to be \textit{complete} if it always gives different outputs for distinct links. In this chapter, we shall a link invariant called the normalised bracket polynomial. It attaches each link a polynomial with coefficients from \(\Z[A, A^{-1}]\), where \(A\) and \(A^{-1}\) are some commuting variables. This invariant is not complete. It was discovered by Louis Kauffman in 1987~\cite{kauffmanstate1987, kauffmaninvariant}. The normalised version of this polynomial is equivalent to the Jones polynomial, which was discovered by Vaughn Jones in 1985 while working on the theory of operator algebras~\cite{jones}. The discovery of the Jones polynomial created a flurry of activity as relations between knot theory and mathematical physics were found. Several generalisations of the Jones/bracket polynomial were immediately found and some long standing problems in knot theory, such as the Tait conjectures were proved. The approaches of Jones and Kauffman are very different while defining their polynomials. While understanding the original route of Jones requires knowledge of the theory of von Neumann algebras, Kauffman takes an elementary, but powerful diagrammatic approach while defining his polynomial. In this chapter, we shall look at Kauffman's bracket polynomial using the so called \textit{state model}, as expounded in his book~\cite{kauffman}.
