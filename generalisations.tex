\chapter{The Jones polynomial and its Generalisations}

% In the previous chapters, we defined Kauffman's bracket polynomial using the state model and then normalised it to get a link invariant. Instead of using the state model, one can define the bracket polynomial in an axiomatic way as well. One defines the polynomial as a function on link diagrams which satisfies certain properties. We would need to check the well-definedness and invariance of such a function under the Reidemeister moves. We thus define the bracket polynomial in the following way.
%
% \begin{defn}[Kauffman's bracket polynomial]
% 	Let \(K\) an unoriented link diagram. Then the bracket \(\bk \in \Z[A, A^{-1}]\) is defined by the rules:
% 	\begin{enumerate}
% 		\item \(\braket<\hspace{-0.5pt}\KPA> = 1\).
% 		\item \(\braket<\KPA \cup K> = (-A^2 - A^{-2}) \braket<K\hspace{1pt}>\).
% 		\item \(\BB = A\BD + A^{-1}\BE\).
% 	\end{enumerate}
% \end{defn}
%
% This definition is consistent with the state model. \(\bk\) is invariant under the type II and type III moves. In order to gain type I invariance, we normalise it by adding an orientation and multiplying \(\bk\) with \((-A^3)^{-w(K)}\).
%
% \subsection{Jones polynomial}
%
% We now give an axiomatic definition of the Jones polynomial.
%
% \[
% F\BF
% G\BG
% H\BH
% I\BI
% J\BJ
% K\BK
% L\BL
% \]
%
% \begin{defn}[Jones polynomial]
% 	Let \(K\) be an oriented link diagram. Then the Jones polynomial \(\displaystyle \V(K) \in \Z\left[t^{1/2}, \frac{1}{t^{1/2}}\right]\) is defined by the rules:
% 	\begin{enumerate}
% 	    \item \(\V\CL = 1\).
% 		\item \(\displaystyle \frac{1}{t}\V\CF - t\V\CG = \left(\sqrt t - \frac{1}{t^{1/2}}\right)\V\CH\).
% 	\end{enumerate}
% \end{defn}
%
% It can be verified that this defines an invariant.




\section{Jones polynomial}

\begin{defn}[Jones polynomial]
	The Jones polynomial invariant \(\V(t, K)\) is defined as \[\V(t, K) \coloneq \operatorname{L}\left(\frac{1}{t^{1/4}}, K\right)\] for an oriented link \(K\).
\end{defn}

Here, \(\displaystyle t = \frac{1}{A^4}\). \(\V(t, K)\) is a Laurent polynomial in \(t^{1/2}\); \(\displaystyle \V(K) \in \Z\left[t^{1/2}, \frac{1}{t^{1/2}}\right]\). With the above definition, we have the following two properties of the Jones polynomial.
\begin{enumerate}
	\item \(\V\left(t, \KPL\right) = 1\).
	\item \(\displaystyle \frac{1}{t}\V\left(t, \KPF\right) - t \V\left(t, \KPG \right) = \left(t^{1/2} - \frac{1}{t^{1/2}}\right)\V\CH\).
\end{enumerate}

We can define the Jones polynomial axiomatically as a Laurent polynomial in \(t^{1/2}\) satisfying the above two properties as well. But then we would have to check well-definedness and invariance under the Reidemeister moves. We derive these properties using the definition. The first property follows evaluating the bracket polynomial on the trivial knot. Orientation of a knot does not matter in this case. Just as with the normalised bracket polynomial, we shall drop the variable in which the polynomial is based if it is clear from the context.

\begin{proof}[Proof of the second property]
	We know that \[\left\langle\KPB\right\rangle = A \left\langle\KPD\right\rangle + B \left\langle\KPE\right\rangle\] and \[\left\langle\KPC\right\rangle = B \left\langle\KPD\right\rangle + A \left\langle\KPE\right\rangle.\] Hence, \[B^{-1}\BB - A^{-1}\BC = \left(\frac{A}{B} - \frac{B}{A}\right)\BE.\] Thus, \[A\BG - A^{-1}\BF = (A^2 - A^{-2})\BH.\] Let \(w = \w\CH\) so that \(\w\CG = w - 1\) and \(\w\CF = w + 1\). Also, let \(\alpha = -A^3\). Then \[A \alpha \BG \alpha^{-w+1} - A^{-1}\alpha^{-1}\BF\alpha^{-w+1} = (A^2 - A^{-2})\BH\alpha^{-w}.\] Thus, \[-A^4 \kau\CF + A^{-4}\kau\CG = (A^2 - A^{-2})\kau\CH.\] Substituting \(A = t^{-1/4}\) yields the desired result.
\end{proof}

The Jones polynomial follows a reversing property.

\begin{thm}[Reversing property]
	Let \(K\) be an oriented link. Let \(K'\) be such that \(K'\) is obtained by reversing the orientation of a component \(K_i \subset K\).

	Let \(\lambda = \lk(K_i, K - K_i)\) denote the total linking number of \(K_i\) with the remaining components of \(K\). Then \[\V(t, K') = t^{-3\lambda} \cdot \V(t, K).\]
\end{thm}

\begin{proof}
	We see that \(\w(K') = \w(K) - 4 \lambda\).
	\begin{align*}
		\kau(A, K') &= (-A^3)^{-\w(K')} \langle K' \rangle\\
		&= (-A^3)^{-\w(K) + 4 \lambda} \langle K \rangle.
	\end{align*}
	Therefore \(\kau(A, K') = (-A^3)^{4\lambda}\kau(A, K)\). Substituting \(A = t^{-1/4}\) yields the desired result.
\end{proof}

\begin{exmp}
	Using the skein relation, we have that \[\frac{1}{t} \V\left(\hspace{-10pt}\KPfigureeight\right) - t\V\Cfigureeightf = \left(t^{1/2} - \frac{1}{t^{1/2}}\right) \V\left(\KPL \,\,\KPL\right).\] But since both \hspace{-13pt}\KPfigureeight and \hspace{-13pt}\KPfigureeightf are trivial knots, \[\V\Cfigureeight = \V\Cfigureeightf = \V\CL = 1.\] Thus, \[\V\left(\KPL \,\,\KPL\right) = \frac{t^{-1} - t}{t^{1/2} - 1/t^{1/2}} = \frac{(t^{-1} - t)(t^{1/2} + 1/t^{1/2})}{t - t^{-1}} = -(t^{1/2} + 1/t^{1/2}) = \updelta.\]
\end{exmp}

The third property is surprising when viewed with respect to the skein relation~\cite{lickorish, mortonjones}. Skein relations of this sort form a general defining feature for various polynomials.

\section{Alexander--Conway polynomial}

\begin{defn}[Alexander--Conway polynomial]
	Let \(K\) be an oriented link diagram. Then the Alexander--Conway polynomial invariant \(\displaystyle \nabla(z, K) \in \Z[z]\) is defined by the rules:
	\begin{enumerate}
		\item \(\nabla\left(z, \KPL\right) = 1\).
		\item \(\displaystyle \nabla\left(z, \KPF\right) - \nabla\left(z, \KPG\right) = \nabla\left(z, \KPH\right)\).
	\end{enumerate}
\end{defn}

This polynomial is a generalisation and reformulation of the original Alexander polynomial.

\section{\textsc{homflypt} polynomial}

The \textsc{homflypt} polynomial was discovered independently by two groups, one group consisting of Jim Hoste, Adrian Ocneanu, Kenneth Millett, Peter Freyd, William Lickorish, and David Yetter and the other group consisting of Polish mathematicians Józef Przytycki and Paweł Traczyk.

\begin{defn}[\textsc{homflypt} polynomial]
	Let \(K\) be an oriented link diagram. Then the \textsc{homflypt} polynomial invariant \(\displaystyle \homfly(\alpha, z, K)\) is defined by the rules:
	\begin{enumerate}
		\item \(\homfly\left(\alpha, z, \KPL\right) = 1\).
		\item \(\displaystyle \alpha \homfly\left(\alpha, z, \KPF\right) - \frac{1}{\alpha} \homfly\left(\alpha, z, \KPG\right) = z \homfly\left(\alpha, z, \KPH\right)\).
	\end{enumerate}
\end{defn}

Note that this polynomial is a two variable polynomial. If we take \(\alpha = t^{-1}\) and \(z = t^{1/2} - 1/t^{1/2}\), we retrieve the Jones polynomial. If we take \(\alpha = 1\), we retrieve the Alexander--Conway polynomial.

\section{Kauffman polynomial}

The Kauffman polynomial \(\operatorname{F}(\alpha, z, K)\) is a semi-oriented two variable polynomial invariant which generalises Kauffman's bracket and the Jones polynomial. This polynomial is a normalisation of a polynomial, \(\kauf(\alpha, z, K)\), defined for unoriented links which satisfies the following properties.
\begin{enumerate}
	\item \(\kauf\left(\alpha, z, \KPL\right) = 1\).
	\item \(\kauf\left(\alpha, z, \KPB\right) + \kauf\left(\alpha, z, \KPC\right) = z \left[\kauf\left(\alpha, z, \KPD\right) + \kauf\left(\alpha, z, \KPE\right)\right]\).
	\item \(\kauf\bigg(\alpha, z, \plusoneu\bigg) = \alpha \kauf\left(\alpha, z, \KPL\right)\)
	\item \(\kauf\bigg(\alpha, z, \minusoneu\bigg) = \frac{1}{\alpha} \kauf\left(\alpha, z, \KPL\right)\).
\end{enumerate}
The Kauffman polynomial is defined by the formula \[\operatorname{F}(\alpha, z, K) = \alpha^{-w(K)} \kauf(\alpha, z, K).\]

