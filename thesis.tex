% !TEX TS-program = lualatex
% !TEX root = ../thesis.tex

\documentclass[10pt, a4paper]{report}

\usepackage[left=1.5in, right=1.5in]{geometry}

\usepackage{setspace}
% \onehalfspacing

\usepackage[toc, page]{appendix}
\providecommand{\appendixname}{Appendices}

\newcommand{\head}[1]{\newpage
	\vspace{3em}
	\begin{center}
		\LARGE{\MakeUppercase{\textbf{#1}}}
	\end{center}
	\vspace{3em}
	\addcontentsline{toc}{chapter}{#1}
}
\renewcommand{\listfigurename}{}
% ------------------------------
% Individual details (Change these!)
% ------------------------------

\newcommand{\thesistitle}{Some topics in knot theory}
\newcommand{\studentname}{Apoorv Potnis}
\newcommand{\studentrollno}{18343}
\newcommand{\advisorname}{Dr.\@ Dheeraj Kulkarni and Prof.\@ Subhash Chaturvedi}
\newcommand{\degreename}{BS-MS}
\newcommand{\subject}{Mathematics}
\newcommand{\thesisdate}{April 2023}

% ------------------------------
% Title page
% ------------------------------

\def\maketitle{
	\begin{titlepage}
		\begin{center}
			\begin{doublespace}
				\textbf{\MakeUppercase{\LARGE{\thesistitle}}} \\
				\ \\
				\ \\
				\normalsize{\textbf{A REPORT}} \\
				\normalsize{\textit{submitted in partial fulfillment of the requirements}} \\
				\normalsize{\textit{for the award of the dual degree of}} \\
				\ \\
				\ \\
				\large{\textbf{Bachelor of Science-Master of Science}} \\
				\normalsize{\textit{in}} \\
				\large{\textbf{\MakeUppercase{\subject}}} \\
				\normalsize{\textit{by}} \\
				\large{\textbf{\MakeUppercase{\studentname}}} \\
				\normalsize{\textbf{(\studentrollno)}} \\
				%\normalsize{\textit{Under the guidance of}} \\
				%\large{\textbf{\MakeUppercase{\advisorname}}}
			\end{doublespace}
			\vfill
			%\centerline{\includegraphics[scale=0.20]{iiser_b.png}}
			\ \\
			\textbf{DEPARTMENT OF \MakeUppercase{\subject}, \\
				INDIAN INSTITUTE OF SCIENCE EDUCATION AND RESEARCH BHOPAL,\\ %Flows onto two lines
				BHOPAL - 462 066} \\
			\ \\
			\textbf{\thesisdate}
		\end{center}
	\end{titlepage}
}














%\usepackage[nomarginpar]{geometry}

% \usepackage[monochrome]{xcolor}

%\usepackage{geometry}
%\geometry{top=1.5in}

\PassOptionsToPackage{math-style=ISO, bold-style=ISO, sans-style=italic, nabla=upright, partial=upright, warnings-off={mathtools-colon,mathtools-overbracket}}{unicode-math}
\usepackage{mathtools}
\usepackage{fontsetup}


%\setmathfont{NewCMsans10-BookOblique}[range=it,script-font=NewCMSans08-BookOblique,sscript-font=NewCMSans08-BookOblique]
%\usepackage{unicode-math, newtx}
% \usepackage{fontspec,unicode-math}\setmainfont[]{STEP}
% \setmathfont[StylisticSet=0]{XITS Math}
% \setmathfont[range={\mathcal,\mathbfcal},StylisticSet=1]{XITS Math}
\usepackage[dvipsnames]{xcolor}
\usepackage{microtype, lipsum, comment, tikz, tikz-3dplot, float, subcaption, tikz-cd, xurl, adjustbox, svg, tkz-euclide}
\usepackage{physics2}\usephysicsmodule{ab,ab.braket}
\usetikzlibrary{knots, hobby, braids, babel, decorations.markings}
% \usepackage{embedall}

%\usepackage[bold=0.04]{xfakebold}
%\usepackage{xfp}

% \usepackage{embedall} \embedfile{thesis.bib}
% \embedfile{apoorv_potnis_synopsis.xmpdata}

\usepackage{dsfont}
% \renewcommand{\symbb}[1]{\setBold \symbfup #1 \unsetBold}
\renewcommand{\symbb}{\mathds}
%\renewcommand{\symbb}{\symbfup}

%\usepackage{kpfonts-otf}

\AtBeginDocument{\newcommand*\pii{\symup{π}}}
\newcommand{\R}{\symbb{R}}
\newcommand{\C}{\symbb{C}}
\renewcommand*{\hbar}{\mathrm{^^^^0127}}
\renewcommand{\i}{\symup{i}}
\newcommand{\I}{\symbb{I}}
\newcommand{\Z}{\symbb{Z}}
\newcommand{\ssep}{\mid}
\newcommand{\B}{\symup{B}}
\newcommand{\GB}{\text{GB}}
\newcommand{\TL}{\text{TL}}
\newcommand{\sigmaa}{\symup{\sigma}}
\newcommand{\tauu}{\symup{\tau}}
\newcommand{\U}{\symup{U}}
\DeclarePairedDelimiterX\presentation[1]\langle\rangle{\def\given{\;\delimsize\vert\;}#1}
\DeclarePairedDelimiter\abs{\lvert}{\rvert}
\DeclarePairedDelimiterX\set[1]\lbrace\rbrace{\setaux#1}
\def\setaux#1|{#1\;\delimsize\vert\;}
\DeclarePairedDelimiter{\norm}{\lVert}{\rVert}
\DeclarePairedDelimiter{\bracket}{\left\langle}{\right\rangle}
\DeclarePairedDelimiter\brak{\left(}{\right)}
\makeatletter
\def\bign#1{\mathclose{\hbox{$\left#1\vbox to8.5\p@{}\right.\n@space$}}\mathopen{}}
\makeatother

%\newcommand{\e}{\symup{e}}
%\newcommand{\M}{\symup{M}}
\newcommand{\upphi}{\symup{\phi}}
\newcommand{\upPhii}{\symup{\Phi}}
\newcommand{\updelta}{\symup{\delta}}
\newcommand{\uprho}{\symup{\rho}}
\DeclareMathOperator{\tr}{tr}
\DeclareMathOperator{\uprhoo}{\uprho}
\DeclareMathOperator{\proj}{\pi}
\DeclareMathOperator{\V}{V}
\DeclareMathOperator{\w}{w}
% \let\P\relax
\DeclareMathOperator{\homfly}{P}
% \newcommand{\V}{\symup{V}}
% \let\L\relax
\DeclareMathOperator{\kau}{L}
\DeclareMathOperator{\lk}{lk}

\newcommand\restr[2]{{% we make the whole thing an ordinary symbol
		\left.\kern-\nulldelimiterspace % automatically resize the bar with \right
		#1 % the function
		\littletaller % pretend it's a little taller at normal size
		\right|_{#2} % this is the delimiter
	}}

\newcommand{\littletaller}{\mathchoice{\vphantom{\big|}}{}{}{}}



%\newcommand{\rthree}{\symbfup{\setBold R\unsetBold}^3}
%\newcommand{\In}{\symbfup{\setBold I\unsetBold}_n}
\newcommand{\rtwo}{\symbb{R}^2}
\newcommand{\sone}{\symbb{S}^1}
\newcommand{\stwo}{\symbb{S}^2}
\newcommand{\sthree}{\symbb{S}^3}
\newcommand{\In}{\symbb{I}_n}
\renewcommand{\I}{\symbb{I}}
\newcommand{\n}{\symbb{N}}
\newcommand{\rthree}{\symbb{R}^3}
\newcommand{\z}{\symbb{Z}}
\newcommand{\bk}{\braket<K\hspace{1pt}>}

\usepackage{xparse}     % <----- for '\NewDocumentCommand' command (necessary)

\NewDocumentCommand{\mybar}{ O{0.60} O{3pt} m }{%  <---- Set the default values here
	\mathrlap{\hspace{#2}\overline{\scalebox{#1}[1]{\phantom{\ensuremath{#3}}}}}\ensuremath{#3}
}


%\usepackage{capt-of}

\usepackage{amsthm}
\theoremstyle{definition}
\newtheorem{thm}{Theorem}
\newtheorem{remark}[thm]{Remark}
\newtheorem{defn}[thm]{Definition}
\newtheorem{prop}[thm]{Proposition}
\newtheorem{exmp}[thm]{Example}

\usepackage[sorting=none]{biblatex}
\addbibresource{thesis.bib}

\usepackage{bookmark, hyperref}
\usepackage[nameinlink]{cleveref}

\crefname{claim}{claim}{claims}
\crefname{prop}{proposition}{propositions}
\crefname{thm}{theorem}{theorems}
\crefname{exmp}{example}{examples}
\crefname{remark}{remark}{remarks}
%\PassOptionsToPackage{sorting=none}{hyperref}
\hypersetup{colorlinks=true, urlcolor=blue, linkcolor=red, citecolor=OliveGreen}
%\usepackage[a-3u]{pdfx}

\author{Apoorv Potnis}
\title{Thesis draft}
\date{\today}

%\usepackage{sansmathfonts}
%\tikzstyle{every picture}+=[font=\sffamily]

\usetikzlibrary{shapes, decorations, arrows.meta, decorations.markings, bending, knots, hobby, spath3, intersections}

\newcommand{\KP}[1]{%
	\begin{tikzpicture}[baseline=-\dimexpr\fontdimen22\textfont2\relax]
		#1
	\end{tikzpicture}%
}
\newcommand{\TREFOIL}{
	\KP{
		\begin{tikzpicture}[xshift=0.6cm, xscale=-0.25, yscale=0.25,
                use Hobby shortcut,
                every trefoil component/.style={line width=0.6pt, draw},
                trefoil component 1/.style={black},
                trefoil component 2/.style={black},
                trefoil component 3/.style={black},
                ]
                \path[spath/save=trefoil] ([closed]90:2) foreach \k in {1,...,3} {
                    .. (-30+\k*240:.5) .. (90+\k*240:2) } (90:2);
				\tikzset{spath/knot={trefoil}{4pt}{1,3,5}}
            \end{tikzpicture}
	}
}
\newcommand{\CWL}{
	\KP{
		\filldraw[fill=none, thick, red] circle (0.7);
		\draw[thick, red] (0.7,0.9) -- (0.7,-0.9);
	}
}
\newcommand{\CWLO}{
	\KP{
		\filldraw[fill=none, thick, red] circle (0.7);
		\draw[thick, red] (0.8,0.9) -- (0.8,-0.9);
	}
}
\newcommand{\CWLI}{
	\KP{
		\filldraw[fill=none, thick, red] circle (0.7);
		\draw[thick, red] (0.6,0.9) -- (0.6,-0.9);
	}
}
\newcommand{\BPB}{
	\KP{
		\draw[line width=0.6pt] (-0.3,0.3) -- (0.3,-0.3);
		\draw[line width=0.6pt] (-0.3,-0.3) -- (-0.05,-0.05);
		\draw[line width=0.6pt] (0.05,0.05) -- (0.3,0.3);
		\draw[line width=0.6pt] (-0.4,-0.31) -- (-0.4,0.31);
		\node at (-0.6,0) {\(\cdots\)};
		\draw[line width=0.6pt] (-0.8,-0.31) -- (-0.8,0.31);
		\draw[line width=0.6pt] (0.4,-0.31) -- (0.4,0.31);
		\node at (0.6,0) {\(\cdots\)};
		\draw[line width=0.6pt] (0.8,-0.31) -- (0.8,0.31);
	}
}
\newcommand{\BPIthree}{
	\KP{
		\draw[line width=0.6pt] (0,-0.3) -- (0,0.3);
		\draw[line width=0.6pt] (0.1,-0.3) -- (0.1,0.3);
		\draw[line width=0.6pt] (0.2,-0.3) -- (0.2,0.3);
	}
}
\newcommand{\BPD}{
	\KP{
		\draw[line width=0.6pt] (-0.3,0.3) .. controls (0,-0.05) .. (0.3,0.3);
		\draw[line width=0.6pt] (-0.3,-0.3) .. controls (0,0.05) .. (0.3,-0.3);
		\draw[line width=0.6pt] (-0.4,-0.31) -- (-0.4,0.31);
		\node at (-0.6,0) {\(\cdots\)};
		\draw[line width=0.6pt] (-0.8,-0.31) -- (-0.8,0.31);
		\draw[line width=0.6pt] (0.4,-0.31) -- (0.4,0.31);
		\node at (0.6,0) {\(\cdots\)};
		\draw[line width=0.6pt] (0.8,-0.31) -- (0.8,0.31);
	}
}
\newcommand{\BPE}{
	\KP{
		\draw[line width=0.6pt] (-0.3,-0.3) .. controls (0.05,0) .. (-0.3,0.3);
		\draw[line width=0.6pt] (0.3,-0.3) .. controls (-0.05,0) .. (0.3,0.3);
		\draw[line width=0.6pt] (-0.4,-0.31) -- (-0.4,0.31);
		\node at (-0.6,0) {\(\cdots\)};
		\draw[line width=0.6pt] (-0.8,-0.31) -- (-0.8,0.31);
		\draw[line width=0.6pt] (0.4,-0.31) -- (0.4,0.31);
		\node at (0.6,0) {\(\cdots\)};
		\draw[line width=0.6pt] (0.8,-0.31) -- (0.8,0.31);
	}
}
\newcommand{\KPA}{%
	\KP{\filldraw[fill=none, line width=0.6pt] circle (0.3);}%
}
\newcommand{\KPB}{%
	\KP{
		\draw[line width=0.6pt] (-0.3,0.3) -- (0.3,-0.3);
		\draw[line width=0.6pt] (-0.3,-0.3) -- (-0.05,-0.05);
		\draw[line width=0.6pt] (0.05,0.05) -- (0.3,0.3);
	}%
}
\newcommand{\KPC}{%
	\KP{
		\draw[line width=0.6pt] (-0.3,-0.3) -- (0.3,0.3);
		\draw[line width=0.6pt] (-0.3,0.3) -- (-0.05,0.05);
		\draw[line width=0.6pt] (0.05,-0.05) -- (0.3,-0.3);
	}%
}
\newcommand{\KPD}{%
	\KP{%
		\draw[line width=0.6pt] (-0.3,0.3) .. controls (0,-0.05) .. (0.3,0.3);
		\draw[line width=0.6pt] (-0.3,-0.3) .. controls (0,0.05) .. (0.3,-0.3);
	}%
}
\newcommand{\KPDonethree}{%
	\KP{%
		\draw[line width=0.6pt] (-0.3,0.3) .. controls (0,-0.05) .. (0.3,0.3);
		\draw[line width=0.6pt] (-0.3,-0.3) .. controls (0,0.05) .. (0.3,-0.3);
		\draw[line width=0.6pt] (0.4,-0.31) -- (0.4,0.31);
	}%
}
\newcommand{\KPE}{%
	\KP{%
		\draw[line width=0.6pt] (-0.3,-0.3) .. controls (0.05,0) .. (-0.3,0.3);
		\draw[line width=0.6pt] (0.3,-0.3) .. controls (-0.05,0) .. (0.3,0.3);
	}%
}
\newcommand{\KPF}{%
	\KP{
		\draw[->, line width=0.6pt] (-0.3,0.3) -- (0.3,-0.3);
		\draw[line width=0.6pt] (-0.3,-0.3) -- (-0.05,-0.05);
		\draw[->, line width=0.6pt] (0.05,0.05) -- (0.3,0.3);
	}%
}
\newcommand{\KPG}{%
	\KP{
		\draw[->, line width=0.6pt] (-0.3,-0.3) -- (0.3,0.3);
		\draw[line width=0.6pt] (-0.3,0.3) -- (-0.05,0.05);
		\draw[->, line width=0.6pt] (0.05,-0.05) -- (0.3,-0.3);
	}%
}
\newcommand{\KPH}{%
	\KP{%
		\draw[->, line width=0.6pt] (-0.3,0.3) .. controls (0,-0.05) .. (0.3,0.3);
		\draw[->, line width=0.6pt] (-0.3,-0.3) .. controls (0,0.05) .. (0.3,-0.3);
	}%
}
\newcommand{\KPI}{%
	\KP{%
		\draw[->, line width=0.6pt] (-0.3,0.3) .. controls (0,-0.05) .. (0.3,0.3);
		\draw[<-, line width=0.6pt] (-0.3,-0.3) .. controls (0,0.05) .. (0.3,-0.3);
	}%
}
\newcommand{\KPJ}{%
	\KP{%
		\draw[->, line width=0.6pt] (-0.3,-0.3) .. controls (0.05,0) .. (-0.3,0.3);
		\draw[->, line width=0.6pt] (0.3,-0.3) .. controls (-0.05,0) .. (0.3,0.3);
	}%
}
\newcommand{\KPK}{%
	\KP{%
		\draw[<-, line width=0.6pt] (-0.3,-0.3) .. controls (0.05,0) .. (-0.3,0.3);
		\draw[->, line width=0.6pt] (0.3,-0.3) .. controls (-0.05,0) .. (0.3,0.3);
	}%
}

\tikzset{%
	arrowat/.style={%
		postaction={decorate,decoration={
				markings,
				mark=at position #1 with {\arrow[xshift=2pt,opacity=1]{>}}}}
	},
}

\newcommand{\KPfigureeight}{
	\KP{
		\begin{knot}[scale=0.3, consider self intersections = true, clip width = 8, ignore endpoint intersections=false]
		    \strand[closed, line width=0.6pt] (0,0) to [out=-90, in=180] (0.5,-1) to [out=0, in=180] (2.5,1) to [out=0, in=90] (3,0) to [out=-90, in=0] (2.5,-1) to [out=180, in=0] (0.5,1) to [out=180, in=90] (0,0);
		\end{knot}
		\draw[<-, line width=0.6pt] (0,0);
	}
}
\newcommand{\KPfigureeightf}{
	\KP{
		\begin{knot}[flip crossing=2, scale=0.3, consider self intersections = true, clip width = 8, ignore endpoint intersections=false]
			\strand[closed, line width=0.6pt] (0,0) to [out=-90, in=180] (0.5,-1) to [out=0, in=180] (2.5,1) to [out=0, in=90] (3,0) to [out=-90, in=0] (2.5,-1) to [out=180, in=0] (0.5,1) to [out=180, in=90] (0,0);
		\end{knot}
		\draw[<-, line width=0.6pt] (0,0);
	}
}




\newcommand{\KPX}[1]{%
	\begin{tikzpicture}[baseline=-\dimexpr\fontdimen22\textfont2\relax]
		#1
	\end{tikzpicture}%
}
\newcommand{\KPL}{%
	\kern -0.06em\KPX{\filldraw[fill=none, line width=0.6pt, decoration={markings, mark=at position 0.25 with {\arrow[xshift=1.5, yshift=0.1\pgflinewidth]{<}}}, postaction={decorate}] circle (0.3);}%
}

\newcommand{\KPS}[1]{%
	\begin{tikzpicture}[baseline=-\dimexpr\fontdimen22\textfont2\relax, scale=0.45]
		#1
	\end{tikzpicture}%
}


\newcommand{\KPSM}{
	\KPS{
		\begin{knot}[consider self intersections=true, ignore endpoint intersections=false,%<- long compilation time, so two paths
			clip width=4,
			clip radius=5pt,looseness=2]
			\strand[line width=0.6pt] (0, 0)  .. controls  +(3, -1) and +(-3,-1) .. (1,0);
		\end{knot}
	}
}

\newcommand{\KPSA}{%
	\kern -0.07em\KPS{\filldraw[fill=none, line width=0.6pt] circle (0.3);}%
}
\newcommand{\KPSB}{%
	\KPS{
		\draw[line width=0.6pt] (-0.3,0.3) -- (0.3,-0.3);
		\draw[line width=0.6pt] (-0.3,-0.3) -- (-0.08,-0.08);
		\draw[line width=0.6pt] (0.08,0.08) -- (0.3,0.3);
	}%
}
\newcommand{\KPSC}{%
	\KPS{
		\draw[line width=0.6pt] (-0.3,-0.3) -- (0.3,0.3);
		\draw[line width=0.6pt] (-0.3,0.3) -- (-0.08,0.08);
		\draw[line width=0.6pt] (0.08,-0.08) -- (0.3,-0.3);
	}%
}
\newcommand{\KPSD}{%
	\KPS{%
		\draw[line width=0.6pt] (-0.3,0.3) .. controls (0,0) .. (0.3,0.3);
		\draw[line width=0.6pt] (-0.3,-0.3) .. controls (0,0) .. (0.3,-0.3);
	}%
}
\newcommand{\KPSE}{%
	\KPS{%
		\draw[line width=0.6pt] (-0.3,-0.3) .. controls (0,0) .. (-0.3,0.3);
		\draw[line width=0.6pt] (0.3,-0.3) .. controls (0,0) .. (0.3,0.3);
	}%
}
\newcommand{\KPSF}{%
	\KPS{
		\draw[->, line width=0.6pt] (-0.3,0.3) -- (0.3,-0.3);
		\draw[line width=0.6pt] (-0.3,-0.3) -- (-0.08,-0.08);
		\draw[->, line width=0.6pt] (0.08,0.08) -- (0.3,0.3);
	}%
}
\newcommand{\KPSG}{%
	\KPS{
		\draw[->, line width=0.6pt] (-0.3,-0.3) -- (0.3,0.3);
		\draw[line width=0.6pt] (-0.3,0.3) -- (-0.08,0.08);
		\draw[->, line width=0.6pt] (0.08,-0.08) -- (0.3,-0.3);
	}%
}
\newcommand{\KPSH}{%
	\KPS{%
		\draw[->, line width=0.6pt] (-0.3,0.3) .. controls (0,-0.0) .. (0.3,0.3);
		\draw[->, line width=0.6pt] (-0.3,-0.3) .. controls (0,0.0) .. (0.3,-0.3);
	}%
}
\newcommand{\KPSI}{%
	\KPS{%
		\draw[->, line width=0.6pt] (-0.3,0.3) .. controls (0,-0.0) .. (0.3,0.3);
		\draw[<-, line width=0.6pt] (-0.3,-0.3) .. controls (0,0.0) .. (0.3,-0.3);
	}%
}
\newcommand{\KPSJ}{%
	\KPS{%
		\draw[->, line width=0.6pt] (-0.3,-0.3) .. controls (0.0,0) .. (-0.3,0.3);
		\draw[->, line width=0.6pt] (0.3,-0.3) .. controls (-0.0,0) .. (0.3,0.3);
	}%
}
\newcommand{\KPSK}{%
	\KPS{%
		\draw[<-, line width=0.6pt] (-0.3,-0.3) .. controls (0.0,0) .. (-0.3,0.3);
		\draw[->, line width=0.6pt] (0.3,-0.3) .. controls (-0.0,0) .. (0.3,0.3);
	}%
}
\newcommand{\KPSS}[1]{%
	\begin{tikzpicture}[baseline=-\dimexpr\fontdimen22\textfont2\relax, scale=0.45]
		#1
	\end{tikzpicture}%
}
\newcommand{\KPSL}{%
	\kern -0.06em\KPSS{\filldraw[fill=none, line width=0.6pt, decoration={markings, mark=at position 0.25 with {\arrow[xshift=1.5, yshift=0.1\pgflinewidth]{<}}}, postaction={decorate}] circle (0.3);}%
}

\newcommand{\KPSX}{%
	\KPS{
		\draw[->, line width=0.6pt] (-0.3,0) -- (0.3,0);
	}%
}
\newcommand{\KPSY}{% 5
	\begin{tikzpicture}[baseline=-\dimexpr\fontdimen22\textfont2\relax, scale=0.45]
		\draw[line width=0.6pt] circle (0.3);
	\end{tikzpicture}%
}

\newcommand{\BA}{\left\langle\KPA\right\rangle}
\newcommand{\BB}{\left\langle\KPB\right\rangle}
\newcommand{\BC}{\left\langle\KPC\right\rangle}
\newcommand{\BD}{\left\langle\KPD\right\rangle}
\newcommand{\BE}{\left\langle\KPE\right\rangle}
\newcommand{\BF}{\left\langle\KPF\right\rangle}
\newcommand{\BG}{\left\langle\KPG\right\rangle}
\newcommand{\BH}{\left\langle\KPH\right\rangle}
\newcommand{\BI}{\left\langle\KPI\right\rangle}
\newcommand{\BJ}{\left\langle\KPJ\right\rangle}
\newcommand{\BK}{\left\langle\KPK\right\rangle}
\newcommand{\BL}{\left\langle\KPL\right\rangle}


\newcommand{\CA}{\left(\KPA\right)}
\newcommand{\CB}{\left(\KPB\right)}
\newcommand{\CC}{\left(\KPC\right)}
\newcommand{\CD}{\left(\KPD\right)}
\newcommand{\CE}{\left(\KPE\right)}
\newcommand{\CF}{\left(\KPF\right)}
\newcommand{\CG}{\left(\KPG\right)}
\newcommand{\CH}{\left(\KPH\right)}
\newcommand{\CI}{\left(\KPI\right)}
\newcommand{\CJ}{\left(\KPJ\right)}
\newcommand{\CK}{\left(\KPK\right)}
\newcommand{\CL}{\left(\KPL\right)}
\newcommand{\Cfigureeight}{\left(\hspace{-10pt}\KPfigureeight\right)}
\newcommand{\Cfigureeightf}{\left(\hspace{-10pt}\KPfigureeightf\right)}
\tikzset{grid/.style={gray,very thin,opacity=1}}


% \usetikzlibrary{decorations.pathreplacing}
% \tikzset{%
% 	show curve controls/.style={
% 		postaction={
% 			decoration={
% 				show path construction,
% 				curveto code={
% 					\draw [blue]
% 					(\tikzinputsegmentfirst) -- (\tikzinputsegmentsupporta)
% 					(\tikzinputsegmentlast) -- (\tikzinputsegmentsupportb);
% 					\fill [red, opacity=0.5]
% 					(\tikzinputsegmentsupporta) circle [radius=.5ex]
% 					(\tikzinputsegmentsupportb) circle [radius=.5ex];
% 				}
% 			},
% 			decorate
% 		}}}

\begin{document}

	\maketitle

	\pagenumbering{roman}

	% ------------------------------
	\head{Certificate}

	This is to certify that {\bf \studentname}, BS-MS (\subject), has worked on the project entitled {\bf `\thesistitle'} under my supervision and guidance. The content of this report is original and has not been submitted elsewhere for the award of any academic or professional degree.

	\vspace{10em}

	\textbf{\thesisdate \hfill Dr.\@ Dheeraj Kulkarni}

	\textbf{IISER Bhopal \hfill Prof.\@ Subhash Chaturvedi}



% 	\textbf{\thesisdate \hfill \advisorname \\ IISER Bhopal}

	\vfill

	\begin{center}
		\begin{tabular}{ccc}
			\textbf{Committee Member} & \textbf{Signature} & \textbf{Date} \\
			\\
			\rule{14em}{0.6pt} & \rule{9em}{0.6pt} & \rule{6em}{0.6pt} \\
			\\
			\rule{14em}{0.6pt} & \rule{9em}{0.6pt} & \rule{6em}{0.6pt} \\
			\\
			\rule{14em}{0.6pt} & \rule{9em}{0.6pt} & \rule{6em}{0.6pt} \\
			\\
			\rule{14em}{0.6pt} & \rule{9em}{0.6pt} & \rule{6em}{0.6pt} \\
		\end{tabular}
	\end{center}

	% ------------------------------
	\begin{doublespace}\head{Academic Integrity and Copyright Disclaimer}\end{doublespace}

	I hereby declare that this project is my own work and, to the best of my knowledge, it contains no materials previously published or written by another
	person, or substantial proportions of material which have been accepted for the award of any other degree or diploma at IISER Bhopal or any other educational institution, except where due acknowledgement is made in the document.

	I certify that all copyrighted material incorporated into this document is in compliance with the Indian Copyright Act (1957) and that I have received written permission from the copyright owners for my use of their work, which is beyond the scope of the law. I agree to indemnify and save harmless IISER Bhopal from any and all claims that may be asserted or that may arise from any copyright violation.

	\vfill

	\textbf{\thesisdate \hfill \studentname \\ IISER Bhopal}

	% ------------------------------
	\head{Acknowledgement}

	Acknowledgement is here.

	\vspace{7em}

	\begin{flushright}
		{\bf \studentname}
	\end{flushright}

	% ------------------------------
	\head{Abstract}

	We shall study some selected topics in knot theory from the book \textit{Knots and Physics}, 4th ed.\@ by Louis Kauffman~\cite{kauffman}. The basics of knot theory have been studied from the book \textit{Knots and Links} by Peter Cromwell~\cite{cromwell}.

	A knot is an embedding of the circle in three-dimensional space. We discuss the state model for bracket polynomial, Jones polynomial and some of its generalisations. One can apply these knot invariants to show that various knots are not ambient isotopic, which is the notion we shall use to distinguish knots. We shall see a resolution of some of the old conjectures in knot theory using the bracket polynomial. Vaughn Jones found the polynomial named after him while he was studying towers of von Neumann algebras. We shall follow a different route following Kauffman to understand the relation amongst braids, links, Temperley--Lieb algebras, which arose in statistical physics, and the Jones algebras.

% 	\begin{figure}
% 		\centering
% 		\begin{tikzpicture}[scale=0.6]
% 			\draw[grid] (-10,-10) grid (10,10);
% 			\begin{knot}[clip width=5, consider self intersections=true,ignore endpoint intersections=false]
% 				\strand[ultra thick, red] (-7,0) to [curve through = {(-6,3) (-4,7) (0,7) (-2,4) (-3,2.5) (-3,2) (0,-4) (-3,-2) (-1,1) (0,3) (-2,8) (2.5,0) (0,-3) (0,-6)}] (-7,0);
% 			\end{knot}
% 		\end{tikzpicture}

% 	\end{figure}

	\tableofcontents
	\pagenumbering{arabic}
% %
% 	\chapter{Introduction to knot theory}

\section{Definition of a knot}

We normally conceive of knots as open strings with `knotted' parts in between. Given a knotted string with two `open' ends, we can simply pull one end of the string to untie it, in the usual way we untie a knot, by inserting an open end into the knotted region is a strategic manner and pulling it on the other side\footnote{There exists another way of unknotting an open knot, often used in magic tricks. One creates another knot using the unknotted portion of the string on one side of the knotted portion such that this new knot `cancels' the original knot. In mathematical terms, the new knot is constructed such that the \textit{connected sum} of the original knot and the new knot gives an unknot. We shall not deal with connected sums of knots in this thesis.}. This way, any knot with two open ends can be untied. But if have a knot in a closed loop, we won't be able to untie it unless we cut it. We want our notion of a knot to be invariant under `pulling'. Thus, we model our mathematical definition on closed loops instead of open. Refer to~\cref{fig:openclosedkots}.

\begin{figure}
    \centering
    \subcaptionbox{A knot with open ends}{
        \begin{tikzpicture}[scale=0.2]
            \begin{knot}[clip width = 10, consider self intersections = true, ignore endpoint intersections = false, use Hobby shortcut]
                \strand[thick, red] (-6,0)
                to (6,0)
                %to [out=0, in=135] (13,-2)
                to [out=0, in=180] (18, -4)
                to [out=0, in=-90] (21,0)
                to [out=90, in=0] (18,4)
                to [out=180, in=0] (9,-4)
                to [out=180, in=-90] (6,0)
                to [out=90, in=180] (9,4)
                to [out=0, in=180] (21,0)
                to (33,0);
                \flipcrossings{1,2,4,5,6,7,8,9}
            \end{knot}
        \end{tikzpicture}}\par\bigskip
    \subcaptionbox{A knot with closed ends}{
        \begin{tikzpicture}[scale=0.2]
            \begin{knot}[clip width = 10, consider self intersections = true, ignore endpoint intersections = false, use Hobby shortcut]
                \strand[thick, red] (13.5,-8) to [out=0, in=180] (10,-8) to [out=180, in=-90] (-3,-6) to [out=90, in=180] (2,0) to (6,0)
                to [out=0, in=180] (18, -4)
                to [out=0, in=-90] (21,0)
                to [out=90, in=0] (18,4)
                to [out=180, in=0] (9,-4)
                to [out=180, in=-90] (6,0)
                to [out=90, in=180] (9,4)
                to [out=0, in=180] (21,0)
                to (25,0) to [out=0, in=90] (30,-6) to [out=-90, in=0] (17,-8) to [out=180, in=0] (13.5,-8);
                \flipcrossings{4,5,7,8,9,10,11,12}
            \end{knot}
        \end{tikzpicture}}
    \caption{Projections of a knot on a plane}
    \label{fig:openclosedkots}
\end{figure}



\begin{defn}[Knot]
    A knot \(K\) is an embedding of \(\sone\) in \(\rthree\).
\end{defn}
Although one can talk about embeddings of higher dimensional `circles' in higher dimensional spaces, we restrict ourselves to knots in the three dimensional space. The above definition turns out out to be too general for our purposes. It takes into consideration certain pathological knots known as \textit{wild knots}. See cref{fig:wildknot}. Although wild knots are an object of study, we won't be dealing with them in this thesis due to their pathological nature. fig shows an example of a wild knot. A section of the knot is scaled down by a constant factor joined on the one side of the previous section. If one does repeats this process infinitely, the section eventually converges to a point, provided that the scaling down is fast enough. We can join the convergence (limit) point to a an end-section on the other side to get wild knot. This knot is continuous everywhere, including the limit point, and an embedding of \(\sone\) in \(\rthree\). Three common ways exist to exclude such behaviour by demanding extra conditions.
\begin{enumerate}
    \item Differentiability.

    We can demand all knots to be differentiable at all points. We can verify visually that all the points, except the limit point of the wild knot are differentiable, or can be made (isotoped) differentiable. The derivative must necessarily change within a section. As the size of the section decreases, the derivative changes more rapidly. This rate of change of the derivative, the double derivative, is a monotonically increasing function as the section number increases. At the limit point, the derivative shall cease to exist by the virtue of `changing too rapidly'. Demanding differentiability forcibly removes the offending limit point. The `wildness' is due to the limit point.

    But this condition comes with a problem as well, namely we cannot use polygons for describing knots.
    \item Piecewise linearity.
    We can demand all the edges of a knot to be piecewise linear. Our knot shall be a polynomial in that case. A polygon has finitely many edges. Infinitely (countably) many sections in a wild knot shall mean infinitely many edges (of decreasing length), which is not allowed. Thus, this condition excludes wild knots.
    \item Local flatness.
\end{enumerate}
In this thesis, we shall take the third route following Cromwell~\cite[chp.~1]{cromwell}. If we consider a local neighbourhood around each point of the knot, except the limit point, then we see visually see that we can always find a small enough local neighbourhood around each such point such that the strand is not `knotted' in that neighbourhood. At the limit point, no matter how small a neighbourhood we take, the strand shall always be `knotted'. We enforce this `local unknottedness' condition by demanding local flatness. Let \(p\) be a point in a knot \(K\), \(\symup{B}(\symup{O}, 1)\) be the unit ball centered at origin \(\symup{O}\) and \(d\) be a diameter of \(\symup{B}(\symup{O}, 1)\).
\begin{defn}[Local flatness]
    The point \(p\) is said to be locally flat if there exists a neighbourhood \(U \ni p\) such that the pair \((U, U \cap K)\) is homeomorphic to \((\symup{B}(\symup{O}, 1), d)\).

    A knot is said to be locally flat if each point in that knot is locally flat. A point that is not locally flat is called wild, and a knot is wild if any of its points are wild.
\end{defn}
Insert figures of a unit ball centered at origin and a diameter, and a section of the knot homeomorphic to the pair.

Consider a spherical neighbourhood around a locally flat point. There exists a radius such that for all neighbourhoods less than this radius, the boundary of the neighbourhood intersects the strand in exactly two points. This is not possible at the limit point in wild knot figure.

\begin{defn}[Tame knots]
    A knot is said to be tame if all its points are locally flat.
\end{defn}

\section{Distinguishing knots}

Any two homeomorphisms of the circle are homeomorphic to the circle and to each other, since being homeomorphic is an equivalence relation. But this means that all knots are homeomorphic to each other. Clearly, homeomorphism is not the correct notion to distinguish knots. When we mean that two knots are distinct, we mean that if we create a physical model of those knots, we cannot `physically deform' one knot into another. \textit{Cutting a knot is not allowed.} One might think that homotopy or isotopy are what we need, but it turns out that the notion of \textit{ambient isotopy} is the correct one.

\begin{defn}[Homotopy]
    A homotopy of a space \(X \subset \rthree\) is a continuous map \(h \colon X \times [0, 1] \rightarrow \rthree\).

    The restriction of \(h\) to level \(t \in [0, 1]\) is \(h_t \colon X \times \{t\} \rightarrow \rthree\). \(h_0\) must be the identity map.
\end{defn}
Note that the continuity of \(h\) implies the continuity of \(h_t\) for all \(t \in [0,1]\). The converse is not true. Insert example here. Homotopy allows a curve to pass through itself. All knots are thus homotopic to the trivial knot. If we do not allow a curve to pass through itself, i.e.\@ if we demand injectivity for each \(h_t\), then we get what is known as an isotopy. But isotopy is not useful for distinguishing knots as well, due to bachelors unknotting. All tame knots, or more generally, all knots with a tame arc turn out to be isotopic to the trivial knot\footnote{See \texttt{\url{https://math.stackexchange.com/a/1336309/239463}}}. It is an open problem if a particular knot known as the Bing sling, which is wild \textit{at all points} is isotopic to the standard unknot\footnote{See \texttt{\url{https://wildandnoncompactknots.wordpress.com/2013/04/30/knots-in-three-space-non-ambient-isotopy-equivalence-classes/}}}.
\begin{prop}[Bachelors' unknotting]
    Every knot with a tame arc is isotopic to the unknot.
\end{prop}
\begin{proof}
%     \begin{figure}
%         \centering
%         \begin{tikzpicture}
%             \begin{knot}[clip width=5]
%                 \strand[thick, red] (-7,0) to [curve through = {(-6,3) (-4,7) (0,7) (0,4.25) (-2,4) (-3,2.5) (-3,2) (0,-4) (-3,-2) (-1,1) (-3,2.5) (3,3) (1,0) (-1,-2) (-2,-4) (0,-6) (0,-8) (0,6) (-6,-3)}] (-7,0);
%             \end{knot}
%         \end{tikzpicture}
%
%     \end{figure}

    Let \(K \subset \rthree\) be a knot with a tame arc.

    Let \(p \in K\) in a tame arc of the knot. Since the knot is locally flat on the arc, by the definition of tameness, we take a ball \(U_p \subset \rthree\) of radius \(\varepsilon\) around the point \(p\) such that the pair \(U_p, U_p \cap K\) is homeomorphic to \((\symup{B}, \symup{d})\), where \(\symup{B}\) is the unit ball in \(\rthree\) centered at the origin and \(\symup{d}\) is the diameter of \(\symup{B}\) along the \(x\)-axis. We choose a parametrization \(f \colon [0, 2\pii) \rightarrow K\) of the knot such that \(f([a, b]) = K \setminus (U_p \cap K) \), where \([a, b] \subset [0, 2\pii)\).

    Let \(r \in \rthree\) be a point outside \(U_p\). Now consider the function \(i_t \colon K \rightarrow \rthree\), defined for each \(t \in [0, 1]\) as follows.
    Let \[\phi_t(a) \coloneq t\left(\frac{a+b}{2}\right) + (1-t)a\] be a family of functions for all \(t \in [0,1]\).
    \begin{enumerate}
        \item If \(f(x) \in U_p\), then \[i_t(f(x)) = f(x).\]
        \item If \(\displaystyle x \in [a, \phi_t(a))\), then \[ i_t(f(x)) = f(a) + \frac{\norm{f(a) - f(\phi_t(a))}}{\norm{a - \phi_t(a)}}(x - a)(r - f(a)).\]
        \item If \(x \in [\phi_t(a), \phi_t(b)]\), then \[i_t(f(x)) = tr + (1-t)f(x).\]
        \item If \(\displaystyle x \in (\phi_t(b), b]\), then \[ i_t(f(x)) = f(b) + \frac{\norm{f(b) - f(\phi_t(b))}}{\norm{b - \phi_t(b)}}(b - x)(r - f(b)).\]
    \end{enumerate}
    Let \(i \colon [0, 1] \times K \rightarrow \rthree\) be a function defined by \(i(t, f(x)) \coloneq i_t (f(x))\).

    \(i\) is defined such that the part inside \(U_p\) is kept unchanged for all \(t\). For \(t =0\), \(i\) does not deform the knot at all. For \(t \in (0, 1)\), the knotted part (in \(\rthree\)) shrinks and the interval in the domain \([a,b]\) which maps to the knotted part also shrinks to \([\phi_t(a), \phi_t(b)]\). This shrinkage of the domain happens linearly with \(t\). All points of the knotted part trace a straight line as they travel under isotopy from their original position to \(r\). Eventually, the knotted part ceases to exist at \(t = 1\) and a single point of the domain, namely, \((a + b)/2\) maps to \(r\).

    In the end, we get a figure consisting of two straight lines meeting at \(r\), and \(U_p \cap K\), the original part of the knot inside \(U_p\). The other endpoints of these lines are \(f(a)\) and \(f(b)\). \(U_p \cap K\) is isotopic to the line joining \(f(a)\) and \(f(b)\). Thus, we get a triangle with points \(r\), \(f(a)\) and \(f(b)\) and we any that any triangle is isotopic to \(\sone\).

    We now prove that \(i\) is continuous. We know that both \(i_t\) and \(i_x\) are continuous and injective for all \(t \in [0, 1]\) and \(x \in [0, 2\pii)\) respectively, where \(i_x\) is defined to be the restriction of \(i\) for a particular \(x\). We also see that \(i_t\) is linear in \(t\). A function continuous in one argument and linear in another is continuous in the product topology. We thus have an isotopy which sends a knot with a tame arc to \(\sone\).
\end{proof}

In the above considered isotopy, we isotoped the set \(X = K\). Instead, we take \(X\) to be the entire space \(\rthree\), or a bounded set which completely covers the knot, then we get the notion of ambient isotopy. This modification ensures that the surrounding space is isotoped as well as we isotope the knot. The knot is a curve which has no volume. If we try bachelors' unknotting on the surrounding space as well, we observe that the surrounding space, which has a finite, non-zero volume cannot shrink to a set of zero volume under isotopy. This finally leads us to the equivalence relation induced by ambient isotopy.

\begin{remark}
    Unless mentioned otherwise, we shall always consider our knots to be tame from now on.
\end{remark}


\begin{defn}[Knot equivalence]
    Two knots \(K_1\) and \(K_2\) are said to be ambient isotopic if there exists an isotopy \(I \colon \rthree \times [0,1] \rightarrow \rthree\) such that \(I(K_1,0) = I_0(K_1) = K_1\) and \(I(K_1,1) = I_1(K_1) = K_2\).
\end{defn}
\begin{prop}
    Knot equivalence is indeed an equivalence relation.
\end{prop}
\begin{proof}
    Reflexivity.
    Symmetry.
    Transitivity.
\end{proof}

\begin{remark}
    Each equivalence class of knots is called a \textit{knot type}. We would often forget the distinction between a knot and its knot type. The intended meaning can be inferred from the context.
\end{remark}

\begin{remark}
    Note that we distinguish between \textit{ambient isotopy} and \textit{isotopy}. Many treatments of knot theory use the word isotopy for ambient isotopy as ambient isotopy is the useful construct in knot theory. Ambient isotopy is an isotopy of the whole space containing the knot, not just the knot.
\end{remark}

Give examples of knots which are ambient isotopic and not ambient isotopic, with images.

\begin{remark}
    In this thesis, we shall look at knot theory in \(\rthree\). One can compactify \(\rthree\) to \(\sthree\) and do knot theory in \(\sthree\), as many treatments do. This does not result in a different knot theory.
\end{remark}

\section{Knot diagrams}

So far, we have depicted knots as curves in a plane (of paper) with the implicit understanding that if a strand goes under another strand, then we break the strand which goes underneath. One defined a knot as an object in three dimensions, but we can project the knot onto a plane to represent it, as we have done so far. We do not lose any topological information if we choose a `nice enough' projection. What we have been doing implicitly can be formalized; we shall see how.

Let \(K\) be a knot and let \(\pi \colon \rthree \rightarrow \rtwo\) be a projection map. A point \(p \in \pi(K)\) is called \textit{regular} if \(\pi^{-1}(p)\) is a single point, and \textit{singular} otherwise. If \(\abs{\pi^{-1}(p)} = 2\), then \(p\) is called a \textit{double} point.


%
%
% \section{Isotopy}
%
% \subsection{Bachelors' unknotting}

% \begin{figure}
%     \centering
% 	\subcaptionbox{}{
% 		\begin{tikzpicture}[scale=0.5]
% 			\begin{knot}[clip width=5, consider self intersections,
% 				ignore endpoint intersections=false]
% 				\strand[thick, red] (7,3) to [curve through={(5,7) .. (2.7,5.2) .. (4,4) ..  (5.6,4.3) .. (6.9,5.5) .. (7,6) .. (6, 8).. (5,5)}] (7.5, 4);
% 				\flipcrossings{1,3}
% 			\end{knot}
%
% 		\end{tikzpicture}
%
% }
% \end{figure}
% .. (6.1,10.3) .. (4.4,10.3) .. (3.2,9.6) .. (4.4,10.3)


% 	\chapter{The Bracket Polynomial}

Given two links or knots, one desires a way to tell if the two links are distinct. Since we can represent knots faithfully on a paper using knot diagrams, one can ask for a way to distinguish knots based on their diagrams. A \textit{link invariant} is a function of a link such that if the evaluation of the function on two links yields different outputs, then the links are distinct, i.e.\@ they are not ambient isotopic to each other. A link invariant is said to be \textit{complete} if it always gives different outputs for distinct links.

In this chapter, we shall a link invariant called the normalised bracket polynomial. It attaches each link a polynomial with coefficients from \(\Z[A, A^{-1}]\), where \(A\) and \(A^{-1}\) are some commuting variables. This invariant is not complete. It was discovered by Louis Kauffman in 1987~\cite{kauffmanstate1987, kauffmaninvariant}. The normalised version of this polynomial is equivalent to the Jones polynomial, which was discovered by Vaughn Jones in 1985 while working on the theory of operator algebras~\cite{jones}. The discovery of the Jones polynomial created a flurry of activity as relations between knot theory and mathematical physics were found. Several generalisations of the Jones/bracket polynomial were immediately found and some long standing problems in knot theory, such as the Tait conjectures were proved. The approaches of Jones and Kauffman are very different while defining their polynomials. While understanding the original route taken by Jones to define his polynomial requires the knowledge of the theory of von Neumann algebras, Kauffman takes an elementary, but powerful diagrammatic approach while defining his polynomial. In this chapter, we shall look at Kauffman's bracket polynomial using the so called \textit{state model}, as expounded in his book~\cite{kauffman}.

\section{State model}

Consider a crossing of an unoriented link as shown in \cref{fig:crossingresolution}. We designate local regions around a crossing a label \(A\) or \(B\) based on the following scheme. Walk along the underpass towards a crossing. The area on the left is assigned the label \(A\) and the area on the right the label \(B\). This assignment is unambiguous. We then `resolve' or `smoothen' or `split' the crossing in two ways, one way which connects the \(A\)-regions and another way which connects the \(B\)-regions. If we resolve a crossing such that \(A\)-regions are connected, then we attach the label \(A\) to the resolved diagram. The same holds for \(B\)-regions. Note that the resolution of the crossing happens only locally, i.e.\@ in an open ball around the crossing which does not intersect other crossings. Such an open ball exists due to the Hausdorff property of the plane.

\begin{figure}
    \centering
	\includesvg{crossingresolution.svg}
	\caption{Resolution of a crossing.}
	\label{fig:crossingresolution}
\end{figure}

We shall do this process recursively for all crossings to get a sets of Jordan curves in a plane with labels attached to them. Refer to \cref{fig:trefoilresolution} where we have carried this process for a trefoil.

\begin{figure}
    \centering
	\includesvg[width=\textwidth]{trefoilresolution}
	\caption{Resolution of all crossings for a trefoil.}
	\label{fig:trefoilresolution}
\end{figure}

Each of the individual diagrams shown in curly brackets in \cref{fig:trefoilresolution} is referred to as a state. To each state are the labels \(A\) and \(B\) attached to it. We can construct the original link unambiguously using the states (and labels attached to them). We shall construct invariants of links by `averaging' over these states.

By averaging, we mean the following. Let \(L\) be a link and \(\sigma\) denote a particular state obtained after resolving all the crossings recursively. We denote the commutative product of the labels associated to that state by \(\braket<K|\sigma>\). For example, \[\left\langle\TREFOIL \Bigg\vert\includesvg[width=0.9cm]{produtoflabels.svg}\right\rangle = A^2B.\] Let \(\norm{\sigma}\) denote one less than the number of loops in \(\sigma\). Thus, for the above state, we have \(\norm{\sigma} = 1-1=0\).

We define the bracket polynomial \(\langle L\rangle\) of a link \(L\) as follows. \[\langle L \rangle = \sum_\sigma\braket<L|\sigma> d^{\norm{\sigma}},\] where \(A\), \(B\) and \(d\) are commuting variables. The bracket polynomial is thus a function of \(A\), \(B\) and \(d\). \(\sigma\) runs over all the states of \(K\).

Thus, we calculate \(\braket<L|s>\) for each state for the trefoil to get
\begin{align*}
    \langle L\rangle &= A^3d^{2-1} + A^2Bd^{1-1} + A^2Bd^{2-1} + A^2Bd^{1-1}\\
					 &\phantom{=\,} + AB^2d^{2-1} + AB^2d^{2-1} + B^3d^{3-1}\\
	\langle L\rangle &= A^3d + 3A^2Bd^0 + 3AB^2d^1 + B^3d^2.
\end{align*}

\begin{thm}[Skein relation]
    \[\BB = A\BD + A^{-1}\BE.\]
\end{thm}

The diagrams in the above theorem should be regarded as local diagrams. We make changes to a diagram locally near the crossing. There exists an open topological ball around each crossing such that the ball does not intersect any other crossing and the diagram remains the same outside this ball.

The proof of the above theorem follows from the definition of \(\braket<L>\) and realizing that the states of a diagram are in one-to-one correspondence with the union of the diagrams with resolved crossings. The above identity is very important and we shall use it very often. For an example of a calculation of the bracket polynomial of a link, please refer to \cref{fig:hopflinkbracket}.

\begin{figure}
    \centering
	\includesvg[width=\textwidth]{hopflinkbracket.svg}
	\caption{Bracket polynomial calculation for the Hopf link}
	\label{fig:hopflinkbracket}
\end{figure}

\begin{thm}
	\[\Bintersection = AB \BE + AB\Bthirty + (A^2 + B^2)\BD.\]
\end{thm}
\begin{proof}
	\[\Bintersection = A\Bigl\langle\hspace{-3pt}\KPthirtyoneone\hspace{-3pt}\Bigr\rangle + B\Bigl\langle\hspace{-3pt}\KPthirtyonetwo\hspace{-3pt}\Bigr\rangle.\] This splits into \[\Bintersection = A\left[A\Bigl\langle\hspace{-2.5pt}\vcenter{\hbox{\KPthirtyonethree}}\hspace{-2.5pt}\Bigr\rangle + B\Bthirty\right] + B\left[A\Bigl\langle\hspace{-2.5pt}\vcenter{\hbox{\KPthirtyonefour}}\hspace{-2.5pt}\Bigr\rangle + B\Bigl\langle\hspace{-2.5pt}\vcenter{\hbox{\KPthirtyonefive}}\hspace{-2.5pt}\Bigr\rangle\right].\] Finally we have \[\Bintersection = AB \BE + AB\Bthirty + (A^2 + B^2)\BD.\]
\end{proof}
It is also clear that \[\Bigl\langle\hspace{-2pt}\plusoneu\Bigr\rangle = (Ad + B)\langle\lineone\rangle\] and \[\Bigl\langle\hspace{-2pt}\minusoneu\Bigr\rangle = (A + Bd)\langle\lineone\rangle.\] We also see that \[\left\langle\KPthirtyonesix\right\rangle = d\langle\lineone\rangle.\]

Let \(B = A^{-1}\) and \(d = -A^2 - A^{-2}\). We shall always use these substitutions from now on.

We see that \[\Bintersection = \BE.\] This illustrates the invariance under a type II move. But note that \[\Bigl\langle\hspace{-2pt}\plusoneu\Bigr\rangle = (-A^{-3})\langle\lineone\rangle\] and \[\Bigl\langle\hspace{-2pt}\minusoneu\Bigr\rangle = (-A^{3})\langle\lineone\rangle.\] This follows from substituting \(B = A^{-1}\) and \(d = -A^2 - A^{-2}\) in \(Ad + B\) to get \(-A^3\). This illustrates that the bracket polynomial is \textit{not} invariant under a type I move. Adding a twist multiplies the bracket by a factor of \(-A^3\) or \(-A^{-3}\), depending upon the type of the twist. This suggests that if we multiply by a compensatory factor, then we can possibly make it invariant under a type I move as well. We shall do this later using a normalisation factor.

Type II invariance of the bracket and the substitutions ensure that the bracket polynomial is invariant under a type III move as well.

\begin{thm}
	\[\Bigl\langle\hspace{-6.5pt}\KPthirtytwoone\hspace{-6.5pt}\Bigr\rangle = \Bigl\langle\hspace{-6.5pt}\KPthirtytwotwo\hspace{-6.5pt}\Bigr\rangle.\]
\end{thm}
\begin{proof}
	We have \[\Bigl\langle\hspace{-6.5pt}\KPthirtytwoone\hspace{-6.5pt}\Bigr\rangle = A\Bigl\langle\hspace{-6.5pt}\KPthirtytwofour\hspace{-6.5pt}\Bigr\rangle + B\Bigl\langle\hspace{-6.5pt}\KPthirtytwofive\hspace{-6.5pt}\Bigr\rangle.\] Using the type II moves, we have \[A\Bigl\langle\hspace{-6.5pt}\KPthirtytwothree\hspace{-6.5pt}\Bigr\rangle + B\Bigl\langle\hspace{-6.5pt}\KPthirtytwosix\hspace{-6.5pt}\Bigr\rangle = \Bigl\langle\hspace{-6.5pt}\KPthirtytwotwo\hspace{-6.5pt}\Bigr\rangle.\]
\end{proof}

	\chapter{The Jones polynomial and its Generalisations}

% In the previous chapters, we defined Kauffman's bracket polynomial using the state model and then normalised it to get a link invariant. Instead of using the state model, one can define the bracket polynomial in an axiomatic way as well. One defines the polynomial as a function on link diagrams which satisfies certain properties. We would need to check the well-definedness and invariance of such a function under the Reidemeister moves. We thus define the bracket polynomial in the following way.
%
% \begin{defn}[Kauffman's bracket polynomial]
% 	Let \(K\) an unoriented link diagram. Then the bracket \(\bk \in \Z[A, A^{-1}]\) is defined by the rules:
% 	\begin{enumerate}
% 		\item \(\braket<\hspace{-0.5pt}\KPA> = 1\).
% 		\item \(\braket<\KPA \cup K> = (-A^2 - A^{-2}) \braket<K\hspace{1pt}>\).
% 		\item \(\BB = A\BD + A^{-1}\BE\).
% 	\end{enumerate}
% \end{defn}
%
% This definition is consistent with the state model. \(\bk\) is invariant under the type II and type III moves. In order to gain type I invariance, we normalise it by adding an orientation and multiplying \(\bk\) with \((-A^3)^{-w(K)}\).
%
% \subsection{Jones polynomial}
%
% We now give an axiomatic definition of the Jones polynomial.
%
% \[
% F\BF
% G\BG
% H\BH
% I\BI
% J\BJ
% K\BK
% L\BL
% \]
%
% \begin{defn}[Jones polynomial]
% 	Let \(K\) be an oriented link diagram. Then the Jones polynomial \(\displaystyle \V(K) \in \Z\left[t^{1/2}, \frac{1}{t^{1/2}}\right]\) is defined by the rules:
% 	\begin{enumerate}
% 	    \item \(\V\CL = 1\).
% 		\item \(\displaystyle \frac{1}{t}\V\CF - t\V\CG = \left(\sqrt t - \frac{1}{t^{1/2}}\right)\V\CH\).
% 	\end{enumerate}
% \end{defn}
%
% It can be verified that this defines an invariant.




\section{Jones polynomial}

\begin{defn}[Jones polynomial]
	The Jones polynomial invariant \(\V(t, K)\) is defined as \[\V(t, K) \coloneq \operatorname{L}\left(\frac{1}{t^{1/4}}, K\right)\] for an oriented link \(K\).
\end{defn}

Here, \(\displaystyle t = \frac{1}{A^4}\). \(\V(t, K)\) is a Laurent polynomial in \(t^{1/2}\); \(\displaystyle \V(K) \in \Z\left[t^{1/2}, \frac{1}{t^{1/2}}\right]\). With the above definition, we have the following two properties of the Jones polynomial.
\begin{enumerate}
	\item \(\V\left(t, \KPL\right) = 1\).
	\item \(\displaystyle \frac{1}{t}\V\left(t, \KPF\right) - t \V\left(t, \KPG \right) = \left(t^{1/2} - \frac{1}{t^{1/2}}\right)\V\CH\).
\end{enumerate}

We can define the Jones polynomial axiomatically as a Laurent polynomial in \(t^{1/2}\) satisfying the above two properties as well. But then we would have to check well-definedness and invariance under the Reidemeister moves. We derive these properties using the definition. The first property follows evaluating the bracket polynomial on the trivial knot. Orientation of a knot does not matter in this case. Just as with the normalised bracket polynomial, we shall drop the variable in which the polynomial is based if it is clear from the context.

\begin{proof}[Proof of the second property]
	We know that \[\left\langle\KPB\right\rangle = A \left\langle\KPD\right\rangle + B \left\langle\KPE\right\rangle\] and \[\left\langle\KPC\right\rangle = B \left\langle\KPD\right\rangle + A \left\langle\KPE\right\rangle.\] Hence, \[B^{-1}\BB - A^{-1}\BC = \left(\frac{A}{B} - \frac{B}{A}\right)\BE.\] Thus, \[A\BG - A^{-1}\BF = (A^2 - A^{-2})\BH.\] Let \(w = \w\CH\) so that \(\w\CG = w - 1\) and \(\w\CF = w + 1\). Also, let \(\alpha = -A^3\). Then \[A \alpha \BG \alpha^{-w+1} - A^{-1}\alpha^{-1}\BF\alpha^{-w+1} = (A^2 - A^{-2})\BH\alpha^{-w}.\] Thus, \[-A^4 \kau\CF + A^{-4}\kau\CG = (A^2 - A^{-2})\kau\CH.\] Substituting \(A = t^{-1/4}\) yields the desired result.
\end{proof}

The Jones polynomial follows a reversing property.

\begin{thm}[Reversing property]
	Let \(K\) be an oriented link. Let \(K'\) be such that \(K'\) is obtained by reversing the orientation of a component \(K_i \subset K\).

	Let \(\lambda = \lk(K_i, K - K_i)\) denote the total linking number of \(K_i\) with the remaining components of \(K\). Then \[\V(t, K') = t^{-3\lambda} \cdot \V(t, K).\]
\end{thm}

\begin{proof}
	We see that \(\w(K') = \w(K) - 4 \lambda\).
	\begin{align*}
		\kau(A, K') &= (-A^3)^{-\w(K')} \langle K' \rangle\\
		&= (-A^3)^{-\w(K) + 4 \lambda} \langle K \rangle.
	\end{align*}
	Therefore \(\kau(A, K') = (-A^3)^{4\lambda}\kau(A, K)\). Substituting \(A = t^{-1/4}\) yields the desired result.
\end{proof}

\begin{exmp}
	Using the skein relation, we have that \[\frac{1}{t} \V\left(\hspace{-10pt}\KPfigureeight\right) - t\V\Cfigureeightf = \left(t^{1/2} - \frac{1}{t^{1/2}}\right) \V\left(\KPL \,\,\KPL\right).\] But since both \hspace{-13pt}\KPfigureeight and \hspace{-13pt}\KPfigureeightf are trivial knots, \[\V\Cfigureeight = \V\Cfigureeightf = \V\CL = 1.\] Thus, \[\V\left(\KPL \,\,\KPL\right) = \frac{t^{-1} - t}{t^{1/2} - 1/t^{1/2}} = \frac{(t^{-1} - t)(t^{1/2} + 1/t^{1/2})}{t - t^{-1}} = -(t^{1/2} + 1/t^{1/2}) = \updelta.\]
\end{exmp}

The third property is surprising when viewed with respect to the skein relation~\cite{lickorish, mortonjones}. Skein relations of this sort form a general defining feature for various polynomials.

\section{Alexander--Conway polynomial}

\begin{defn}[Alexander--Conway polynomial]
	Let \(K\) be an oriented link diagram. Then the Alexander--Conway polynomial invariant \(\displaystyle \nabla(z, K) \in \Z[z]\) is defined by the rules:
	\begin{enumerate}
		\item \(\nabla\left(z, \KPL\right) = 1\).
		\item \(\displaystyle \nabla\left(z, \KPF\right) - \nabla\left(z, \KPG\right) = \nabla\left(z, \KPH\right)\).
	\end{enumerate}
\end{defn}

This polynomial is a generalisation and reformulation of the original Alexander polynomial.

\section{\textsc{homflypt} polynomial}

The \textsc{homflypt} polynomial was discovered independently by two groups, one group consisting of Jim Hoste, Adrian Ocneanu, Kenneth Millett, Peter Freyd, William Lickorish, and David Yetter and the other group consisting of Polish mathematicians Józef Przytycki and Paweł Traczyk.

\begin{defn}[\textsc{homflypt} polynomial]
	Let \(K\) be an oriented link diagram. Then the \textsc{homflypt} polynomial invariant \(\displaystyle \homfly(\alpha, z, K)\) is defined by the rules:
	\begin{enumerate}
		\item \(\homfly\left(\alpha, z, \KPL\right) = 1\).
		\item \(\displaystyle \alpha \homfly\left(\alpha, z, \KPF\right) - \frac{1}{\alpha} \homfly\left(\alpha, z, \KPG\right) = z \homfly\left(\alpha, z, \KPH\right)\).
	\end{enumerate}
\end{defn}


% 	\chapter{Braids and the Jones polynomial}

\section{Motivation}

\adjustbox{scale=0.9,center}{
\begin{tikzcd}
	\centering
	& & {\begin{array}{@{}c@{}}\text{Jones} \\ \text{algebra}\end{array}} \arrow{r}{\text{\normalsize Markov}}[swap]{\text{\normalsize trace}} &[5em] {\begin{array}{@{}c@{}}\text{Jones} \\ \text{polynomial}\end{array}} \\
	{\begin{array}{@{}c@{}}\text{Oriented}\\ \text{link}\end{array}} \arrow[r] & {\begin{array}{@{}c@{}}\text{Artin} \\ \text{braid} \\ \text{group}\end{array}} \arrow[rd, sloped, "\text{Repres.}"] \arrow[ru, sloped, "\text{Repres.}"] & & \\
	& & {\begin{array}{@{}c@{}}\text{Temperley--} \\ \text{Lieb} \\ \text{algebra}\end{array}} \arrow{r}{\text{\normalsize Diagrammatic}}[swap]{\text{\normalsize trace}} &[5em] {\begin{array}{@{}c@{}}\text{Normalised}\\ \text{bracket}\\ \text{polynomial}\end{array}} \arrow[uu, leftrightarrow, sloped, above, "\text{\normalsize Substitution}"]
\end{tikzcd}}\vspace{10pt}

As remarked earlier, Jones arrived at his polynomial indirectly while working on the theory of operator algebras~\cite{jones}. In his course of investigations, he constructed a tower of algebras nested in one another with the property that each of these algebras is generated by a set of generators satisfying a particular set of relations. A degree of similarity between these relations and the relations among the generators of the \textit{Artin braid group} was pointed out to Jones by a student during a seminar~\cite[p.~216]{cromwell}, which led to the investigations of Jones into knot theory. Jones had defined a notion of a \textit{trace} on his algebras; more specifically a trace function obeying the \textit{Markov property}. As we shall see, one can express every link in terms of a (non-unique) \textit{braid}. Jones then defined a representation of such a braid into his algebras. The trace of an algebra representation of a braid, which is in turn obtained from the link, can be calculated. The Jones polynomial was realized as such a trace.

In this chapter, we shall not travel the original route of Jones to reach his polynomial as it requires the knowledge of the theory of von Neumann algebras. Instead, we shall follow the approach described by Kauffman in his book to construct a representation of the Artin braid group into the \textit{Temperley--Lieb algebra}~\cite[chp.~8]{kauffman}. These algebras admit a diagrammatic intrepretation and our definition of a trace on these algebras shall be diagrammatic in nature as well. Via this trace, we eventually reach the bracket polynomial, which we already know to be equivalent to the Jones polynomial as demonstrated earlier. The Jones algebra can be recovered from the Temperley--Lieb algebra by a choice of substitutions. The Temperley--Lieb algebra arose during the study of certain statistical models in physics~\cite{temperley-lieb}. This algebra can be viewed as a sub-algebra of a broader framework of the \textit{partition algebra}~\cite{partition-algebra}.

\section{Geometric representation of braids}

\subsection{Definition}

\begin{figure}
	\centering
	\begin{tikzpicture}
		[scale=1.5,
		cube/.style={thick,black},
		grid/.style={very thin,gray},
		axis/.style={->,blue,thick}]

		%draw a grid in the x-y plane
		%\foreach \x in {-0.5,0,...,2.5}
		%\foreach \y in {-0.5,0,...,2.5}
		%{
		%	\draw[grid] (\x,-0.5) -- (\x,2.5);
		%	\draw[grid] (-0.5,\y) -- (2.5,\y);
		%}

		%draw the axes
		\draw[axis] (0,0,0) -- (6,0,0) node[anchor=west]{\(x\)};
		\draw[axis] (0,0,0) -- (0,3,0) node[anchor=south]{\(y\)};
		\draw[axis] (0,0,0) -- (0,0,2) node[anchor=north east]{\(z\)};

		%draw the top and bottom of the cube
		\draw[cube] (0,0,-1) -- (0,2,-1) -- (5,2,-1) -- (5,0,-1) -- cycle;


		%draw the edges of the cube
		\draw[cube] (0,0,-1) -- (0,0,1);
		\draw[cube] (0,2,-1) -- (0,2,1);
		\draw[cube] (5,0,-1) -- (5,0,1);
		\draw[cube] (5,2,-1) -- (5,2,1);
% 		\draw[thick, black] (4,0,-1) -- (5,0,-1) -- (5,0,1) -- (4,0,1);

		\begin{knot}[clip width=8]
			\strand[thick, red] (1,0,0) to [in angle=-90, out angle=90, curve through = {(1.5,1,1)}] (4,2,0) node[black, below]{\(p_1\)};
			\strand[thick, red] (2,0,0) to [in angle=-90, out angle=90, curve through = {(1.7,1.5,0.5)}] (1,2,0) node[black, below]{\(p_2\)};
			\strand[thick, red] (3,0,0) to [in angle=-90, out angle=90, curve through = {(3,0.5,0.7)}] (2,2,0) node[black, below]{\(p_3\)};
			\strand[thick, red] (4,0,0) to [in angle=-90, out angle=90, curve through = {(3,1.3,-0.5)}] (3,2,0) node[black, below]{\(p_4\)};
			\flipcrossings{2}
		\end{knot}

		\node[below, label={\(q_1\)}] at (1,2,0) {};
		\node[below, label={\(q_2\)}] at (2,2,0) {};
		\node[below, label={\(q_3\)}] at (3,2,0) {};
		\node[below, label={\(q_4\)}] at (4,2,0) {};

		\draw[cube] (0,0,+1) -- (0,2,+1) -- (5,2,+1) -- (5,0,+1) -- cycle;

	\end{tikzpicture}
	\captionof{figure}{Three dimensional geometric representation of braids.}
	\label{fig:3drepbraids}
\end{figure}

We shall now understand some basics of braid theory. Emil Artin introduced the Artin braid group explicitly~\cite{artinzopfe, artin, friedman}.

An \(n\)-braid is an element of the Artin braid group \(\B_n\), defined via the following presentation on the generators \(\sigmaa_i\), for \(1 \leq i \leq n-1 \).
\[\B_n \coloneq\Biggl\langle
\begin{array}{c|rcll}
	& \sigmaa_i \sigmaa_i^{-1} &=& \symbb{I}_n^{\symrm{a}} &\\
	\sigmaa_1,\ldots, \sigmaa_{n-1} & \sigmaa_i \sigmaa_{i+1} \sigmaa_i &=& \sigmaa_{i+1} \sigmaa_i \sigmaa_{i+1} &\\
	& \sigmaa_i \sigmaa_j &=& \sigmaa_j \sigmaa_i &\text{if } \abs{i-j} \geq 2
\end{array}
\Biggr\rangle,\] where \(\symbb{I}_n^{\symrm{a}}\) is the identity of \(\B_n\).
Thus, \(\B_n\) is the quotient of the free group of \(n-1\) generators with the smallest normal subgroup of the free group containing the elements \(\sigmaa_i \sigmaa_i^{-1}\), \(\sigmaa_i \sigmaa_{i+1} \sigmaa_i \sigmaa_{i+1}^{-1} \sigmaa_i^{-1} \sigmaa_{i+1}^{-1}\) and \(\sigmaa_i \sigmaa_j \sigmaa_i^{-1} \sigmaa_{j}^{-1}\) with appropriately restricted \(i\) and \(j\). We want to recover this algebraic definition using the intuitive understanding of braids that we have. For, that we shall now see a geometric construction in the three dimensional Euclidean space to represent the Artin braid group. This shall make clear the geometric interpretation of the relations as well.

Consider two \textit{ordered} sets of points \(L_1 \coloneq \{p_1 \coloneq (1,0,0),\ldots, p_n \coloneq (n,0,0)\}\) and \(L_2 \coloneq \{q_1 \coloneq (1,1,0),\ldots, q_n \coloneq (n,1,0)\}\) as shown in \cref{fig:3drepbraids} for \(n=4\). Elements of \(L_1\) are called bottom points and elements of \(L_2\) are called the top points. For \(1 \leq i \leq n\), consider a family of non-intersecting continuous curves \(\gamma_i \colon [0, 1] \rightarrow \rthree\) such that
\begin{enumerate}
	\item \(\gamma_i (0) = p_i\) and \(\gamma_i (1) = q_j\) for \(1 \leq i, j \leq n\).
	\item Any plane perpendicular to the \(xy\)-plane and parallel to the \(x\)-axis intersects each of the curves either exactly once or not at all.
	\item All the curves lie in the cube determined by the vertices \((0,0,1)\), \((0,0,-1)\), \((0,1,1)\), \((0,1,-1)\), \((n+1,0,1)\), \((n+1,0,-1)\), \((n+1,1,1)\), \((n+1,1,-1)\).
\end{enumerate}
Such a labelled curve is called a strand in standard position and a family of such labelled \(n\) stands is called an \(n\)-strand set in a standard position. We can ambient isotope or rigidly move an \(n\)-strand set to get another \(n\)-strand set, possibly not in a standard position. Two \(n\)-strand sets are said to be equivalent if they are related by a sequence of rigid motions of the strand sets, and ambient isotopies of the strand sets such that the space outside the cube, along with the endpoints, remains fixed. We shall refer to an equivalence class of such \(n\)-strands as a geometric \(n\)-braid. Thus, a geometric \(n\)-braid is well-defined.

\begin{remark}
	Even though we have restricted our strands to the a bounded cube in the standard position, we can in principle change the bounds of our cube in \(x\) and \(z\) directions to any value and get the same theory. We shall not pursue this approach here.
\end{remark}

\subsection{Standard projection}

We call the projection of a standard position \(n\)-strand set onto the \(xy\)-plane to be a two dimensional representation of a braid. Such a projection is drawn in \cref{fig:2drepbraids}. It should be noted that a standard position \(n\)-strand set is unique only up to ambient isotopy, thus correspondingly the two dimensional representation of such a set is also unique only up to ambient isotopy, namely the ambient isotopies of the projection of the cube and the ambient isotopies such that the projection is a two dimensional representation of a braid for all times.

\begin{figure}
	\centering
	\begin{tikzpicture}
		[scale=1.5,
		cube/.style={thick,black},
		grid/.style={very thin,gray},
		axis/.style={->,blue,thick}]

		%draw a grid in the x-y plane
		%\foreach \x in {-0.5,0,...,2.5}
		%\foreach \y in {-0.5,0,...,2.5}
		%{
		%	\draw[grid] (\x,-0.5) -- (\x,2.5);
		%	\draw[grid] (-0.5,\y) -- (2.5,\y);
		%}

		%draw the axes
		\draw[axis] (0,0,0) -- (5,0) node[anchor=west]{\(x\)};
		\draw[axis] (0,0,0) -- (0,2.8) node[anchor=south]{\(y\)};

		\begin{knot}[clip width=8]
			\strand[thick, red] (1,0) to [in angle=-90, out angle=90, curve through = {(1.5,1)}] (4,2) node[black, below]{\(p_1\)};
			\strand[thick, red] (2,0) to [in angle=-90, out angle=90, curve through = {(1.7,1.5)}] (1,2) node[black, below]{\(p_2\)};
			\strand[thick, red] (3,0) to [in angle=-90, out angle=90, curve through = {(3,0.5)}] (2,2) node[black, below]{\(p_3\)};
			\strand[thick, red] (4,0) to [in angle=-90, out angle=90, curve through = {(3,1.3)}] (3,2) node[black, below]{\(p_4\)};
			\flipcrossings{2}
		\end{knot}

		\node[below, label={\(q_1\)}] at (1,2) {};
		\node[below, label={\(q_2\)}] at (2,2) {};
		\node[below, label={\(q_3\)}] at (3,2) {};
		\node[below, label={\(q_4\)}] at (4,2) {};
	\end{tikzpicture}
	\captionof{figure}{Two dimensional geometric representation of braids.}
	\label{fig:2drepbraids}
\end{figure}

Now onwards, we shall always visually represent geometric \(n\)-braids using their standard two dimensional projections.

\subsection{Group structure}

Multiplication of any two \(n\)-geometric braids \(b_1\) and \(b_2\), denoted by \(b_1 b_2\) is defined as follows (\cref{fig:braidmultiplication}). Ambient isotope and then rigidly move \(b_1\) and \(b_2\) \textit{separately} in the standard position. Now translate only \(b_1\) in the \(+y\) direction by unit distance. The bottom points of \(b_1\) and the top points of \(b_2\) now coincide. Concatenate their strands and shrink the concatenated strands in the \(y\) direction by half keeping fixed the bottom points of \(b_2\). The result is another geometric \(n\)-braid \(b_1 b_2\) in the standard position. Multiplication defined this way is associative.

\begin{figure}
	\centering
	\subcaptionbox[t]{\(b_1\)}{
		\begin{tikzpicture}
			[scale=1,
			cube/.style={thick,black},
			grid/.style={very thin,gray},
			axis/.style={->,blue,thick}]

			%draw a grid in the x-y plane
			%\foreach \x in {-0.5,0,...,2.5}
			%\foreach \y in {-0.5,0,...,2.5}
			%{
			%	\draw[grid] (\x,-0.5) -- (\x,2.5);
			%	\draw[grid] (-0.5,\y) -- (2.5,\y);
			%}

			%draw the axes
			\draw[axis] (0,0) -- (4,0) node[anchor=west]{\(x\)};
			\draw[axis] (0,0) -- (0,2.8) node[anchor=south]{\(y\)};

			\begin{knot}[clip width=5]
				\strand[thick, red] (1,0) to [in angle=-90, out angle=90, curve through = {(1.5,0.5)}] (1,2) node[black, below]{\(p_1\)\strut};
				\strand[thick, red] (2,0) to [in angle=-90, out angle=90, curve through = {(1.9,0.5)}] (3,2) node[black, below]{\(p_2\)\strut};
				\strand[thick, red] (3,0) to [in angle=-90, out angle=90, curve through = {(2,1.3)}] (2,2) node[black, below]{\(p_3\)\strut};
				\flipcrossings{2}
			\end{knot}

			%\node[below, label={\(q_1\)}] at (1,2) {};
			%\node[below, label={\(q_2\)}] at (2,2) {};
			%\node[below, label={\(q_3\)}] at (3,2) {};

			%\node at (2, -1.1) {\(b_1\)};
		\end{tikzpicture}\label{fig:braidmultiplication1}}\qquad
	\subcaptionbox[t]{\(b_2\)}{
		\begin{tikzpicture}
			[scale=1,
			cube/.style={thick,black},
			grid/.style={very thin,gray},
			axis/.style={->,blue,thick}]

			%draw a grid in the x-y plane
			%\foreach \x in {-0.5,0,...,2.5}
			%\foreach \y in {-0.5,0,...,2.5}
			%{
			%	\draw[grid] (\x,-0.5) -- (\x,2.5);
			%	\draw[grid] (-0.5,\y) -- (2.5,\y);
			%}

			%draw the axes
			\draw[axis] (0,0) -- (4,0) node[anchor=west]{\(x\)};
			\draw[axis] (0,0) -- (0,2.8) node[anchor=south]{\(y\)};

			\begin{knot}[clip width=5]
				\strand[thick, red] (1,0) to [in angle=-90, out angle=90, curve through = {(1.7,1)}] (2,2) node[black, below]{\(p_1'\)\strut};
				\strand[thick, red] (2,0) to [in angle=-90, out angle=90, curve through = {(1.7,0.5)}] (1,2) node[black, below]{\(p_2'\)\strut};
				\strand[thick, red] (3,0) to [in angle=-90, out angle=90, curve through = {(3,0.5)}] (3,2) node[black, below]{\(p_3'\)\strut};
			\end{knot}

			%\node[below, label={\(q_1'\)}] at (1,2) {};
			%\node[below, label={\(q_2'\)}] at (2,2) {};
			%\node[below, label={\(q_3'\)}] at (3,2) {};

			%\node at (2, -1.2) {\(b_2\)};
		\end{tikzpicture}\label{fig:braidmultiplication2}}
	\subcaptionbox[t]{\(b_1 b_2\)}{
		\begin{tikzpicture}
			[scale=1,
			cube/.style={thick,black},
			grid/.style={very thin,gray},
			axis/.style={->,blue,thick}]

			%draw a grid in the x-y plane
			%\foreach \x in {-0.5,0,...,2.5}
			%\foreach \y in {-0.5,0,...,2.5}
			%{
			%	\draw[grid] (\x,-0.5) -- (\x,2.5);
			%	\draw[grid] (-0.5,\y) -- (2.5,\y);
			%}

			%draw the axes
			\draw[axis] (0,0) -- (4,0) node[anchor=west]{\(x\)};
			\draw[axis] (0,0) -- (0,5) node[anchor=south]{\(y\)};

			\begin{knot}[clip width=5]
				\strand[thick, red] (1,0) to [in angle=-90, out angle=90, curve through = {(1.7,1)}] (2,2) node[black, below]{\(p_1''\)\strut};
				\strand[thick, red] (2,0) to [in angle=-90, out angle=90, curve through = {(1.7,0.5)}] (1,2) node[black, below]{\(p_2''\)\strut};
				\strand[thick, red] (3,0) to [in angle=-90, out angle=90, curve through = {(3,0.5)}] (3,2) node[black, below]{\(p_3''\)\strut};
				\strand[thick, red] (1,2) to [in angle=-90, out angle=90, curve through = {(1.5,2.5)}] (1,4);
				\strand[thick, red] (2,2) to [in angle=-90, out angle=90, curve through = {(1.9,2.5)}] (3,4);
				\strand[thick, red] (3,2) to [in angle=-90, out angle=90, curve through = {(2,3.3)}] (2,4);
				%\strand[thick, red] (1,0) to [in angle=-90, out angle=90, curve through = {(2,1)}] (3,4) node[black, below]{\(p_1\)};
				%\strand[thick, red] (2,0) to [in angle=-90, out angle=90, curve through = {(1,1)}] (1,4) node[black, below]{\(p_2\)};
				%\strand[thick, red] (3,0) to [in angle=-90, out angle=90, curve through = {(3,1)}] (2,4) node[black, below]{\(p_3\)};
				%\strand[thick, red] (1,2) to [in angle=-90, curve through = {(1.7,2)}] (2,4);
				%\strand[thick, red] (2,2) to [in angle=-90, curve through = {(1.7,1.5)}] (1,4) node[black, below];
				%\strand[thick, red] (3,2) to [in angle=-90, curve through = {(3,1.5)}] (3,4) node[black, below];
			\end{knot}

			%\node[below, label={\(q_1''\)}] at (1,4) {};
			%\node[below, label={\(q_2''\)}] at (2,4) {};
			%\node[below, label={\(q_3''\)}] at (3,4) {};

			%\node at (2, -1.2) {\(b_1 b_2\)};
		\end{tikzpicture}\label{fig:braidmultiplication3}}
	\caption{Multiplication of two braids (before shrinking).}\label{fig:braidmultiplication}
\end{figure}

We shall now drop the axes as well while representing two dimensional geometric \(n\)-braids.

An \(n\)-strand set such that each \(\gamma_i\) is a straight line segment connecting the \(i^\text{th}\) bottom point to the \(i^\text{th}\) top point is called the identity geometric \(n\)-braid and is denoted by \(\In\) (\cref{fig:braididentity}).

\begin{figure}\centering
	\begin{tikzpicture}
		[scale=0.7,
		cube/.style={thick,black},
		grid/.style={very thin,gray},
		axis/.style={->,blue,thick}]

		%draw a grid in the x-y plane
		%\foreach \x in {-0.5,0,...,2.5}
		%\foreach \y in {-0.5,0,...,2.5}
		%{
		%	\draw[grid] (\x,-0.5) -- (\x,2.5);
		%	\draw[grid] (-0.5,\y) -- (2.5,\y);
		%}

		%draw the axes
		\begin{knot}[clip width=5]
			\strand[thick, red] (1,0) to (1,2);
			\strand[thick, red] (2,0) to (2,2);
			\strand[thick, red] (3,0) to (3,2);
			\flipcrossings{2}
		\end{knot}

		\node[below, label={\(q_1\)}] at (1,2) {};
		\node[below, label={\(q_2\)}] at (2,2) {};
		\node[below, label={\(q_3\)}] at (3,2) {};
		\node[below, label={\(p_1\)}] at (1,-0.7) {};
		\node[below, label={\(p_2\)}] at (2,-0.7) {};
		\node[below, label={\(p_3\)}] at (3,-0.7) {};
	\end{tikzpicture}
	\captionof{figure}{The identity \(\I_3\)}
	\label{fig:braididentity}
\end{figure}

A geometric \(n\)-braid \(a\) such that \(ab = ba = \In\) for some geometric \(n\)-braid \(b\) is called the inverse of \(b\) and denoted is by \(b^{-1}\). We shall see that each element has an inverse.

With these operations, the set of geometric \(n\)-braids becomes a group, which we shall denote by \(\GB_n\).

\subsection{Generators}

By the virtue of ambient isotopy, we can move the crossings in a two dimensional representation of a geometric \(n\)-braid such that each crossing lies in a region bounded by two lines parallel to the \(x\)-axis. Moreover, we can arrange the crossings such that each such region contains only one crossing. Thus, if we give the information regarding the type of each crossing for each such region, we can faithfully reconstruct the two dimensional representation. To this end, we define the generators of a geometric \(n\)-braid.

Denote by \(\tauu_i\) the geometric \(n\)-braid such that
\begin{enumerate}
	\item \(\gamma_i (1) = q_{i+1}\), \(\gamma_{i+1} (1) = q_{i}\), and \(\gamma_j (1) = q_j\) when \(j\) does not equal \(i\) or \(i+1\).
	\item \(\pii_{xy}(\gamma_i(t)) \geq 0\) and \(\pii_{xy}(\gamma_{i+1}(t)) \leq 0\) for all \(t \in [0,1]\).
\end{enumerate}
\(\pii_{xy}\) is the projection maps onto to \(xy\)-plane. \(\tauu_1,\ldots, \tauu_{n-1}\) are the generators of \(\GB_n\) (\cref{fig:geometricbraidgenerators}).

\begin{figure}\centering
	\subcaptionbox{\(\tauu_i\)}{
		\begin{tikzpicture}
			[scale=0.7,
			cube/.style={thick,black},
			grid/.style={very thin,gray},
			axis/.style={->,blue,thick}]

			%draw a grid in the x-y plane
			%\foreach \x in {-0.5,0,...,2.5}
			%\foreach \y in {-0.5,0,...,2.5}
			%{
			%	\draw[grid] (\x,-0.5) -- (\x,2.5);
			%	\draw[grid] (-0.5,\y) -- (2.5,\y);
			%}

			%draw the axes
			\begin{knot}[clip width=8]
				\strand[thick, red] (-3,-0.1) to (-3,2.1);
				\strand[thick, red] (-1,-0.1) to (-1,2.1);
				\strand[thick, red] (0,0) to (2,2);
				\strand[thick, red] (2,0) to (0,2);
				\strand[thick, red] (3,-0.1) to (3,2.1);
				\strand[thick, red] (5,-0.1) to (5,2.1);
			\end{knot}
			\node[ultra thick, red] at (-2,1) {\(\cdots\)};
			\node[ultra thick, red] at (4,1) {\(\cdots\)};

			\node[below, label={\(p_1\)}] at (-3,-0.7) {};
			\node[below, label={\(p_{i-1}\)}] at (-1,-0.7) {};
			\node[below, label={\(q_1\)}] at (-3,2) {};
			\node[below, label={\(q_{i-1}\)}] at (-1,2) {};
			\node[below, label={\(q_i\)}] at (0,2) {};
			\node[below, label={\(q_{i+1}\)}] at (2,2) {};
			\node[below, label={\(p_i\)}] at (0,-0.7) {};
			\node[below, label={\(p_{i+1}\)}] at (2,-0.7) {};
			\node[below, label={\(p_{i+2}\)}] at (3,-0.7) {};
			\node[below, label={\(p_{n}\)}] at (5,-0.7) {};
			\node[below, label={\(q_{i+2}\)}] at (3,2) {};
			\node[below, label={\(q_{n}\)}] at (5,2) {};

			%\node at (1, -1.1) {\(\tauu_i\)};
		\end{tikzpicture}}
	\qquad\quad\subcaptionbox{\(\tauu_i^{-1}\)}{
		\begin{tikzpicture}
			[scale=0.7,
			cube/.style={thick,black},
			grid/.style={very thin,gray},
			axis/.style={->,blue,thick}]

			%draw a grid in the x-y plane
			%\foreach \x in {-0.5,0,...,2.5}
			%\foreach \y in {-0.5,0,...,2.5}
			%{
			%	\draw[grid] (\x,-0.5) -- (\x,2.5);
			%	\draw[grid] (-0.5,\y) -- (2.5,\y);
			%}

			%draw the axes
			\begin{knot}[clip width=8]
				\strand[thick, red] (-3,-0.1) to (-3,2.1);
				\strand[thick, red] (-1,-0.1) to (-1,2.1);
				\strand[thick, red] (0,0) to (2,2);
				\strand[thick, red] (2,0) to (0,2);
				\strand[thick, red] (3,-0.1) to (3,2.1);
				\strand[thick, red] (5,-0.1) to (5,2.1);
				\flipcrossings{1}
			\end{knot}
			\node[ultra thick, red] at (-2,1) {\(\cdots\)};
			\node[ultra thick, red] at (4,1) {\(\cdots\)};

			\node[below, label={\(p_1\)}] at (-3,-0.7) {};
			\node[below, label={\(p_{i-1}\)}] at (-1,-0.7) {};
			\node[below, label={\(q_1\)}] at (-3,2) {};
			\node[below, label={\(q_{i-1}\)}] at (-1,2) {};
			\node[below, label={\(q_i\)}] at (0,2) {};
			\node[below, label={\(q_{i+1}\)}] at (2,2) {};
			\node[below, label={\(p_i\)}] at (0,-0.7) {};
			\node[below, label={\(p_{i+1}\)}] at (2,-0.7) {};
			\node[below, label={\(p_{i+2}\)}] at (3,-0.7) {};
			\node[below, label={\(p_{n}\)}] at (5,-0.7) {};
			\node[below, label={\(q_{i+2}\)}] at (3,2) {};
			\node[below, label={\(q_{n}\)}] at (5,2) {};


			%\node at (1, -1.1) {\(\tauu_i^{-1}\)};
		\end{tikzpicture}}
	\caption{Generators \(\tauu_i\) and \(\tauu_i^{-1}\)}
	\label{fig:geometricbraidgenerators}
\end{figure}

\noindent For example, in \cref{fig:braidmultiplication} we have \(b_1 = \tauu_2 \in \GB_3\), \(b_2 = \tauu_1 \in \GB_3\) and \(b_1 b_2 = \tauu_2 \tauu_1 \in \GB_3\).

If we multiply \(\tauu_i\) and \(\tauu_i^{-1}\) to form \(\tauu_i \tauu_i^{-1}\), we observe that \(\tauu_i \tauu_i^{-1} = \I_n\), where \(\tauu_i, \tauu_i^{-1} \in \B_n\) for all \(n \geq 2\) (\cref{fig:type2}).

\begin{figure}\centering
	\begin{tikzpicture}
		[scale=0.7]

		\begin{knot}[clip width=5]
			\strand[thick, red] (0,0) to [in angle=-90, out angle=90, curve through = {(2,2)}] (0,4);
			\strand[thick, red] (2,0) to [in angle=-90, out angle=90, curve through = {(0,2)}] (2,4);
			\flipcrossings{1,2}
		\end{knot}

		\node at (1, -0.7) {\(\tauu_i \tauu_i^{-1}\)};
	\end{tikzpicture}
	\quad\quad\quad\quad
	\begin{tikzpicture}
		[scale=0.7]

		\begin{knot}[clip width=5]
			\strand[thick, red] (0,0) to (0,4);
			\strand[thick, red] (2,0) to (2,4);
			\flipcrossings{1,2}
		\end{knot}

		\node at (1, -0.7) {\(\I_n\)};
	\end{tikzpicture}

	\captionof{figure}{A type II move illustrating \(\tauu_i \tauu_i^{-1} = \I_n\).}
	\label{fig:type2}
\end{figure}

We can perform a move equivalent to the type III move to see that \(\tauu_i \tauu_{i+1} \tauu_i = \tauu_{i+1} \tauu_i \tauu_{i+1}\) (\cref{fig:type3}).

\begin{figure}\centering
	\begin{tikzpicture}
		[scale=0.6]

		\begin{knot}[clip width=5]
			\strand[ultra thick, blue] (0,0) to [in angle=-90, out angle=90, curve through = {(2,2) (4,4)}] (4,6);
			\strand[thick, red] (2,0) to [in angle=-90, out angle=90, curve through = {(0,2) (0,4)}] (2,6);
			\strand[thick, red] (4,0) to [in angle=-90, out angle=90, curve through = {(4,2) (2,4)}] (0,6);
		\end{knot}

		\node at (2, -0.7) {\(\tauu_i \tauu_{i+1} \tauu_i\)};
	\end{tikzpicture}
	\quad\quad\quad\quad
	\begin{tikzpicture}
		[scale=0.6]

		\begin{knot}[clip width=5]
			\strand[ultra thick, blue] (0,0) to [in angle=-90, out angle=90, curve through = {(0,2) (2,4)}] (4,6);
			\strand[thick, red] (2,0) to [in angle=-90, out angle=90, curve through = {(4,2) (2,4)}] (2,6);
			\strand[thick, red] (4,0) to [in angle=-90, out angle=90, curve through = {(2,2) (0,4)}] (0,6);
			%\flipcrossings{1,2}
		\end{knot}

		\node at (2, -0.7) {\(\tauu_{i+1} \tauu_i \tauu_{i+1}\)};
	\end{tikzpicture}

	\captionof{figure}{A type III move illustrating \(\tauu_i \tauu_{i+1} \tauu_i = \tauu_{i+1} \tauu_i \tauu_{i+1}\).}
	\label{fig:type3}
\end{figure}

We can slide two crossings vertically across each other if this does not change the ambient isotopy type. This is possible if the two crossings we wish to slide do not share a strand. This gives us the relation \(\tauu_i \tauu_j = \tauu_j \tauu_i\) if \(\abs{i - j} \geq 2\) (\cref{fig:sliding}).

\begin{figure}\centering
	\begin{tikzpicture}
		[scale=0.5]

		\begin{knot}[clip width=5]
			\strand[thick, red] (0,0) to [in angle=-90, out angle=90, curve through = {(2,2)}] (2,6);
			\strand[thick, red] (2,0) to [in angle=-90, out angle=90, curve through = {(0,2)}] (0,6);
			\strand[thick, red] (4,0) to [in angle=-90, out angle=90, curve through = {(4,4)}] (6,6);
			\strand[thick, red] (6,0) to [in angle=-90, out angle=90, curve through = {(6,4)}] (4,6);
		\end{knot}

		\node at (3, -0.7) {\(\tauu_i \tauu_j\)};
	\end{tikzpicture}
	\quad\quad\quad\quad
	\begin{tikzpicture}
		[scale=0.5]

		\begin{knot}[clip width=5]
			\strand[thick, red] (0,0) to [in angle=-90, out angle=90, curve through = {(0,4)}] (2,6);
			\strand[thick, red] (2,0) to [in angle=-90, out angle=90, curve through = {(2,4)}] (0,6);
			\strand[thick, red] (4,0) to [in angle=-90, out angle=90, curve through = {(6,2)}] (6,6);
			\strand[thick, red] (6,0) to [in angle=-90, out angle=90, curve through = {(4,2)}] (4,6);
			%\flipcrossings{1,2}
		\end{knot}

		\node at (3, -0.7) {\(\tauu_j \tauu_i\)};
	\end{tikzpicture}

	\captionof{figure}{Sliding of crossings illustrating \(\tauu_i \tauu_j = \tauu_j \tauu_i\).}
	\label{fig:sliding}
\end{figure}

Let \(w\) be a word of length \(m\) in \(\GB_n\); \(w = \prod_{j=1}^{m} \tauu^{\pm 1}_{\alpha_j}\) where \(1 \leq \alpha_j \leq n-1\). Every element of \(\GB_n\) and \(\B_n\) can be expressed as a product of its generators, albeit non-uniquely. We define a homomorphism
\begin{align*}
	\upPhi &\colon \GB_n \rightarrow \B_n,\\
	\upPhi &\colon \prod_{j=1}^{m} \tauu^{\pm 1}_{\alpha_j} \mapsto \prod_{j=1}^{m} \sigmaa^{\pm 1}_{\alpha_j} \text{ for all } m \in \symbb{N}.
\end{align*}

\noindent We can see that \(\upPhi\) is a surjection as follows. Take an element \(\prod_{j=1}^{m} \sigmaa^{\pm 1}_{\alpha_j} \in \B_n\), \(\upPhi\) maps \(\prod_{j=1}^{m} \tauu^{\pm 1}_{\alpha_j}\) to \(\prod_{j=1}^{m}\sigmaa^{\pm 1}_{\alpha_j}\). Proving that \(\upPhi\) is an injection is harder and a proof can be found in ~\cite[chp.~2]{murasugibraids}, and~\cite{bohnenblust}.
\begin{thm}
    \(\upPhi\) is an isomorphism, i.e.\@ \(\B_n\) and \(\GB_n\) are isomorphic.
\end{thm}
\noindent This allows us to forget the distinction between \(\B_n\) and \(\GB_n\).

\section{Closure of braids}

We define the closure of a geometric \(n\)-braid as follows. Consider a geometric \(n\)-braid in the standard position. For each \(1 \leq i \leq n\), we construct the following sequence of line connected line segments. Join \((i, 1, 0)\), \((i, i, 0)\), \((i, i, 0)\), \((i, -i, 0)\), \((i, -i, 0)\), \((i, 0, 0)\) consecutively. We then join \(\gamma_i\) to the constructed line segments. Repeating this process for all \(i\) gives the closure of a braid (\cref{fig:closure}). We denote the closure of a geometric \(n\)-braid \(b\) by \(\overline{b}\). Closure of a braid is unique up to ambient isotopy. Two equivalent braid words have the same closures, thus making the closure well-defined.

\begin{figure}\centering
	\subcaptionbox{\(b\)}{
		\begin{tikzpicture}
			[scale=0.6]

			\begin{knot}[clip width=5]
				\strand[very thick, blue] (0,0) to[in angle=-90, out angle=90, curve through = {(1,1)}] (2,2);
				\strand[very thick, blue] (2,0) to[in angle=-90, out angle=90, curve through = {(1,1)}] (0,2);
				\strand[very thick, blue] (4,0) to (4,2);
				%\strand[thick, red] (6,0) to [in angle=-90, out angle=90, curve through = {(6,4)}] (4,6);
			\end{knot}

			%\node at (2, -0.7) {\(b\)};
		\end{tikzpicture}}
	\quad\quad\subcaptionbox{\(\overline{b}\)}{
		\begin{tikzpicture}
			[scale=0.6]

			\begin{knot}[clip width=5]
				\strand[very thick, blue] (0,0) to[in angle=-90, out angle=90, curve through = {(1,1)}] (2,2);
				\strand[very thick, blue] (2,0) to[in angle=-90, out angle=90, curve through = {(1,1)}] (0,2);
				\strand[very thick, blue] (4,0) to (4,2);
				\strand[thick, red] (4,2) to (4,3) to (5,3) to (5,-1) to (4,-1) to (4,0);
				\strand[thick, red] (2,2) to (2,4) to (6,4) to (6,-2) to (2,-2) to (2,0);
				\strand[thick, red] (0,2) to (0,5) to (7,5) to (7,-3) to (0,-3) to (0,0);
				%\strand[thick, red] (0,4) t
				%\flipcrossings{1,2}
			\end{knot}

			%\node at (5, -6.7) {\(\overline{b}\)};
		\end{tikzpicture}}
	\caption{Closure of a braid with \(b = \tauu_1 \in \GB_2\).}
	\label{fig:closure}
\end{figure}

\begin{prop}
	Every closure of a geometric \(n\)-braid is a link.
\end{prop}
\begin{proof}[Proof]
	The outer line segments are locally flat by virtue of being piecewise linear. Due to the second condition regarding the intersection of a plane in the standard representation of geometric \(n\)-braid, we can project the strand onto the \(y\)-axis and this projection would be a homeomorphism. We take an \(\epsilon\) neighbourhood, \(N(\epsilon)\) around the strand. The pair \((N(\epsilon), \gamma_i)\) would then be homeomorphic to \((\symup{B}, \symup{B} \cap x\text{-axis})\), where \(\symup{B}\) is the three-dimensional unit ball around the origin. We have local flatness at the end points as well due to the union of two locally flat curves.
\end{proof}

\begin{remark}
	Suppose \(\tauu_i \in \GB_n\) and \(\tauu_i' \in \GB_{m}\) are generators where \(n < m\). Then the closures of these two generators are not ambient isotopic. The closure of the latter contains one more non-linking loop.
\end{remark}
\subsection{Alexander Theorem}

The theorems of James Alexander~\cite{alexander} and Andrei Markov Jr.~\cite{markov} relate braids to knots.

\begin{thm}[Alexander]
	Every link is ambient isotopic to the closure of a geometric braid, for some \(n \in \symbb{N}\).
\end{thm}
\begin{proof}
	Consider a piecewise linear, regular projection \(\symup{\pi}(L)\) of a link \(L\) on a plane. We choose a point \(O\) in the projection plane which is not collinear with any of the line segments. This can be done since a the link has only finitely many line segments. Let \(P \in \symup{\pi}(L)\). The vector \(OP\) can move either clockwise or anti-clockwise as \(P\) moves along the link projection. We wish to modify the line segments such that \(OP\) moves in only one sense, say anti-clockwise, as \(P\) moves along the entire length of the link projection. We now fix our attention on a line segment corresponding to a clockwise rotation. We divide the segment into sub-parts such that each part shares at-most one crossing point with other line segments. If \(A\) and \(B\) are end-points of such a line segment, then we may replace this line segment with two another line segments \(AC\) and \(CB\), such that \(C\) is another point not on belonging to \(\symup{\pi}(L)\) and the triangle \(ABC\) encloses \(O\). If \(AB\) originally passed under (or over) a line segment of \(\symup{\pi}(L)\), then the modified line segments \(AC\) and \(CB\) must pass under (or over) of the line segments of \(\symup{\pi}(L)\) as well. This move shall not change the link type as it shall be a combination of sliding, type 2 and type 3 moves. In the resulting triangle, we have two orientations possible, one path which travels via \(C\) and the other path which does not. The vector \(OP\) shall move in the opposite, anti-clockwise sense while traversing from \(A\) to \(B\) via \(C\), instead of \(AB\). We can repeat this process for all of the (finitely many) line segments which turn clockwise. In the end, we obtain a projection such that \(OP\) moves in only the anti-clockwise sense, as \(P\) moves along the entire length of the link projection. We can ambient isotope the projection such that all the crossings lie in the projection of a cube, more precisely the cube constructed while defining a geometric braid. The end-points can be made to match as well. The above procedure of triangular moves shall guarantee the monotonicity that is required.
\end{proof}

Henceforth, unless specified otherwise, we shall always work with the projections of the standard representation of a geometric \(n\)-braid and its closure and refer to these projections simply as a braid and its closure.

\subsection{Conjugation}

If \(b, g \in \GB_n\), we observe that \(\overline{g b g^{-1}}\) is ambient isotopic to \(\overline{b}\) (\cref{fig:conjugation}). The upper strands in \(g\) and \(g^{-1}\) are connected via the closure strand. We can slide \(g\) and \(g^{-1}\) via the closure strands to the `other side' of \(b\) to annihilate each other. This can be achieved by a type II move.

\begin{figure}
	\centering
	\subcaptionbox{}{
		\begin{tikzpicture}
			\begin{knot}[clip width = 5]
				\strand[thick, red] (0,0) to [in=-90, out=90](1,1);
				\strand[thick, red] (0,2) to [out=90, in=-90](1,3) to (1,3.5) to (1.5,3.5) to (1.5,-0.5) to (1,-0.5) to (1,0);
				\strand[thick, red] (1,0) to [in=-90, out=90](0,1);
				\strand[thick, red] (1,2) to [out=90, in=-90](0,3) to (0,4) to (2,4) to (2,-1) to (0,-1) to (0,0);
				\strand[very thick, blue] (1,1) to [out=90,in=-90, blue] (0,2);
				\strand[very thick, blue] (0,1) to [out=90,in=-90, blue] (1,2);
				\flipcrossings{2}
			\end{knot}
			\node at (-0.5,0.5) {\(g^{-1}\)};
			\node at (-0.5,1.5) {\(b\)};
			\node at (-0.5,2.5) {\(g\)};
		\end{tikzpicture}}
	\quad\quad\subcaptionbox{}{
		\begin{tikzpicture}
			\begin{knot}[clip width = 5]
				\strand[thick, red] (0,2) to (0,3.5) to (1.5,3.5) to (2.5,2.5) to (3,2.5) to (3,0.5) to (2.5,0.5) to (1.5,-0.5) to (0,-0.5) to (0,1);
				\strand[thick, red] (1,2) to (1,2.5) to (1.5,2.5) to (2.5,3.5) to (3.5,3.5) to (3.5,-0.5) to (2.5,-0.5) to (1.5,0.5) to (1,0.5) to (1,1);
				\strand[very thick, blue] (1,1) to [out=90,in=-90, blue] (0,2);
				\strand[very thick, blue] (0,1) to [out=90,in=-90, blue] (1,2);
				\flipcrossings{1,2}
			\end{knot}
			\node at (2,-1) {\(g^{-1}\)};
			\node at (-0.5,1.5) {\(b\)};
			\node at (2,4) {\(g\)};
		\end{tikzpicture}}\par\bigskip
	\subcaptionbox{}{
		\begin{tikzpicture}
			\begin{knot}[clip width = 5]
				\strand[thick, red] (0,2) to (0,3.75) to (2.5,3.75) to (2.5,2.75) to (1.5,1.75) to (1.5,1.25) to (2.5,0.25) to (2.5,-0.75) to (0,-0.75) to (0,1);
				\strand[thick, red] (1,2) to (1,3.25) to (1.5,3.25) to (1.5,2.75) to (2.5,1.75) to (2.5,1.25) to (1.5,0.25) to (1.5,-0.25) to (1,-0.25) to (1,1);
				\strand[very thick, blue] (1,1) to [out=90,in=-90, blue] (0,2);
				\strand[very thick, blue] (0,1) to [out=90,in=-90, blue] (1,2);
				\flipcrossings{1,2}
			\end{knot}
			\node at (3,0.75) {\(g^{-1}\)};
			\node at (-0.5,1.5) {\(b\)};
			\node at (3,2.25) {\(g\)};
		\end{tikzpicture}}
	\quad\quad\subcaptionbox{}{
		\begin{tikzpicture}
			\begin{knot}[clip width = 5]
				\strand[thick, red] (0,2) to (0,3) to (2,3) to (2,0) to (0,0) to (0,1);
				\strand[thick, red] (1,2) to (1,2.5) to (1.5,2.5) to (1.5,0.5) to (1,0.5) to (1,1);
				\strand[very thick, blue] (1,1) to [out=90,in=-90, blue] (0,2);
				\strand[very thick, blue] (0,1) to [out=90,in=-90, blue] (1,2);
				\flipcrossings{1,2}
			\end{knot}
			\node at (-0.5,1.5) {\(b\)};
		\end{tikzpicture}}
	\caption{Conjugation process illustrating the link equivalence of \(\overline{g b g^{-1}}\) and \(\overline{b}\), with \(b = \tauu_1^{-1} \in \GB_1\) and \(g = \tauu_1^{-1} \in \GB_1\).}
	\label{fig:conjugation}
\end{figure}

\begin{remark}
	The closures of conjugate braids are ambient isotopic as links. The braids themselves are not.
\end{remark}

\subsection{Markov move}

We observe that if \(b \in \GB_n\), then \(b \tauu_n \in \GB_{n+1}\), \(b \tauu_n^{-1} \in \GB_{n+1}\) and \(b\) have ambient isotopic closures (\cref{fig:markovmove}), although \(b\), \(b\tauu_n\) and \(b\tauu_n^{-1}\) are not equivalent as braids. That is, we can add a strand and a crossing of that strand with another strand without changing the link type (of the closure). We can also remove a strand and a crossing if that strand does not cross any other strand. We visually see that adding the above mentioned strands anywhere between the existing strands is equivalent to adding the strands on the right.

\begin{figure}
	\centering
	\subcaptionbox{\(\overline{b}\)}{
		\begin{tikzpicture}
			\begin{knot}[clip width = 5]
				\strand[thick, red] (0,2) to (0,3) to (2,3) to (2,0) to (0,0) to (0,1);
				\strand[thick, red] (1,2) to (1,2.5) to (1.5,2.5) to (1.5,0.5) to (1,0.5) to (1,1);
				\strand[very thick, blue] (1,1) to [out=90,in=-90, blue] (0,2);
				\strand[very thick, blue] (0,1) to [out=90,in=-90, blue] (1,2);
				\flipcrossings{2}
			\end{knot}
			\node at (-0.5,1.5) {\(b\)};
		\end{tikzpicture}}
	\quad\quad\subcaptionbox{\(\overline{b\tauu_2}\)}{
		\begin{tikzpicture}
			\begin{knot}[clip width = 5]
				\strand[very thick, blue] (0,0) to (0,1) to [out=90,in=-90, blue] (1,2);
				\strand[thick, red] (1,2) to (1,3) to (3,3) to (3,-1) to (1,-1) to (1,0);
				\strand[very thick, blue] (1,0) to [out=90,in=-90, blue] (2,1) to (2,2);
				\strand[thick, red] (2,2) to (2,2.5) to (2.5,2.5) to (2.5,-0.5) to (2,-0.5) to (2,0);
				\strand[very thick, blue] (2,0) to [out=90,in=-90, blue]  (1,1) to [out=90,in=-90, blue] (0,2);
				\strand[thick, red] (0,2) to (0,3.5) to (3.5,3.5) to (3.5,-1.5) to (0,-1.5) to (0,0);
				\flipcrossings{1}
			\end{knot}
		\end{tikzpicture}}
	\caption{Markov move with \(b = \tauu_1^{-1}\).}
	\label{fig:markovmove}
\end{figure}

\subsection{Markov Theorem}

\begin{thm}[Markov]
	Two braids whose closures are ambient isotopic to each other are related by a finite sequence of the following operations.
	\begin{enumerate}
		\item Braid equivalences, i.e.\@ equivalences resulting due to the braid relations.
		\item Conjugation.
		\item Markov moves.
	\end{enumerate}
\end{thm}
\noindent A proof of the above theorem, based on the notes of J.\@ H.\@ Roberts and the seminar of an unknown speaker at Princeton University, can be found in the book of Joan Birman~\cite[chp.~2]{birman}. Markov gave a sketch of the proof in 1936~\cite{markov}. A shorter proof may be found in Morton's paper~\cite{morton}.

\section{Orientation}

We see visually that the result of a Markov move on a braid is equivalent to performing a type I move on the braid closure. We know that a type I move increases or decreases the writhe of a link by a unit value. Since all the crossings in a braid closure occur only in the cube containing the braid strands, we can define the writhe of a braid equal to the writhe of the braid closure by assigning each braid crossing a value, either \(+1\) or \(-1\). Our assignment must be consistent with our earlier assignment for knots. But for this procedure, we need to assign an orientation to the braid. We assign all the strands (inside the braid cube) a downward orientation. Doing so, we see that \(\tauu_i\) inherits \(+1\) value while \(\tauu_i^{-1}\) inherits \(-1\). We could instead have assigned all the strands an upward orientation as well. This would not have changed the values of \(\tauu_i\) and \(\tauu_i^{-1}\) (\cref{fig:writhebraids}). What is not allowed is assigning arbitrary orientation to strands. If we assign the orientation arbitrarily, then the well-definedness of the orientation cannot be guaranteed. Two distinct strands in a braid could be connected via the closure strands and one would need to check the whole connected link component of the braid closure for a well-defined closure.

\noindent Thus, the writhe \(w\) of the braid \(b\) with a word representation \(\prod_{j=1}^{m} \tauu^{\beta_j}_{\alpha_j}\) of length \(m\), where \(\beta_j \in \{+1, -1\}\) is \[w(b) = \sum_{j = 1}^{m}\beta_j.\] Note that the writhe of a braid is dependent on its word representation. We also see that \(w(b) = w(\overline{b})\).

\begin{figure}
	\centering
	\subcaptionbox{}{
		\begin{tikzpicture}
			\begin{knot}[clip width = 5]
				\strand[<-, thick, red] (1,1) to (0,2);
				\strand[<-, thick, red] (0,1) to (1,2);
				\flipcrossings{1}
			\end{knot}
		\end{tikzpicture}}\label{a}\quad
	\subcaptionbox{}{
		\begin{tikzpicture}
			\begin{knot}[clip width = 5]
				\strand[->, thick, red] (1,1) to (0,2);
				\strand[->, thick, red] (0,1) to (1,2);
				\flipcrossings{1}
			\end{knot}
		\end{tikzpicture}}\label{b}\quad
	\subcaptionbox{}{
		\begin{tikzpicture}
			\begin{knot}[clip width = 5]
				\strand[<-, thick, red] (1,1) to (0,2);
				\strand[<-, thick, red] (0,1) to (1,2);
			\end{knot}
		\end{tikzpicture}}\label{c}\quad
	\subcaptionbox{}{
		\begin{tikzpicture}
			\begin{knot}[clip width = 5]
				\strand[->, thick, red] (1,1) to (0,2);
				\strand[->, thick, red] (0,1) to (1,2);
			\end{knot}
		\end{tikzpicture}}\label{d}
	\caption{Assignment of crossing values. (a) Downward orientation for \(\tauu_i\) corresponding to \(+1\). (b) Upward orientation for \(\tauu_i\) corresponding to \(+1\). (c) Downward orientation for \(\tauu_i^{-1}\) corresponding to \(-1\). (d) Upward orientation for \(\tauu_i^{-1}\) corresponding to \(-1\).}
	\label{fig:writhebraids}
\end{figure}

\section{Markov trace}

We won't distinguish between \(\B_n\) and \(\GB_n\) from now on. We can now move finally towards the Jones/bracket polynomial with the information we have. If we have a function \(J_n \colon \B_n \rightarrow R\), where \(R\) is a commutative ring, then using the Markov Theorem we can construct link invariants from the family of functions \(\{J_n\}\) if the following conditions are satisfied.
\begin{enumerate}
    \item \(J_n\) is well defined. \(J_n(b)\) = \(J_n(b')\) if \(b = b'\).
	\item \(J_n(b) = J_n(gbg^{-1})\) if \(g, b \in \B_n\).
	\item If \(b \in \B_n\), then there exists a constant \(\alpha\in R\), independent of \(n\), such that \[J_{n+1} (b\sigma_n) = \alpha^{+1} J_n(b)\] and \[J_{n+1} (b\sigma_{n}^{-1}) = \alpha^{-1} J_n(b).\]
\end{enumerate}
\noindent The last condition reminds us of the normalisation needed in order to make the bracket polynomial invariant under the type I move. Its purpose here is the same.

A family of functions \(\{J_n\}\) satisfying the above given three conditions is called a Markov trace on \(\{\B_n\}\). For any link \(L\) which is ambient isotopic to \(\overline{b}\), where \(b \in \B_n\), we define \(J(L) \in R\) as follows. \[J(L) \coloneq \alpha^{-w(b)}J_n(b).\] We call \(J(L)\) the link invariant for the Markov trace \(\{J_n\}\).

\begin{thm}
    \(J\) is an invariant of ambient isotopy for oriented links.
\end{thm}
\begin{proof}
	Suppose \(L \sim \overline{b}\) and \(L'' \sim \overline{b'}\), where \(\sim\) denotes the ambient isotopy relation. By the Markov Theorem, we can obtain \(\overline{b'}\) via an application of a finite sequence of the moves mentioned in the Markov Theorem  on \(\overline{b'}\). Each such move leaves \(J\) invariant. \(J_n\) is already invariant under braid equivalences and conjugation by definition. The \(\alpha^{-w(b)}\) factor cancels the effect of a type I move.
\end{proof}

We can define the bracket polynomial for braids in the same way as we did for links. Define
\begin{align*}
	\langle \cdot \rangle &\colon \B_n \rightarrow \z[A, A^{-1}]\\
	\langle \cdot \rangle &\colon b \mapsto \langle \overline{b}\rangle.
\end{align*}
We simply evaluate the bracket on the closure of the braid. We observe that this function is a Markov trace with \(\alpha = -A^{3}\). We know that \[\BB = A\BD + A^{-1}\BE.\] In terms of braids, we have \[\left\langle\BPB\right\rangle = A\left\langle\BPD\right\rangle + A^{-1}\left\langle\BPE\right\rangle.\] But \(\BPE\) is the identity braid. Thus, if we denote \(\BPD\) by \(\symup{U}_i\), we can write \[\left\langle{\sigmaa_{i}^{-1}}\right\rangle = A\braket<\symup{U}_i> + A^{-1}\braket<\I_n>.\] Similarly, one could write \[\braket<\sigmaa_i> = A\braket<\I_n> + A^{-1}\braket<\symup{U}_i>.\] Note that \(\symup{U}_i\) does not belong to the braid group and is a new object. We refer to them as ``input-output forms'' or as ``hooks''. They are cup and cap combinations involving the \(i\)-th and \((i+1)\)-th strands (\cref{fig:inputoutputforms}).
\begin{figure}
    \centering
	\subcaptionbox{\(U_1\)}{
		\begin{tikzpicture}[scale=2]
			\draw[thick, red] (-0.3,0.3) .. controls (-0.3,-0.06) and (0.3,-0.06) .. (0.3,0.3);
			\draw[thick, red] (-0.3,-0.3) .. controls (-0.3,0.06) and (0.3,0.06) .. (0.3,-0.3);
			\draw[thick, red] (0.5,-0.3) -- (0.5,0.3);
			\draw[thick, red] (0.7,-0.3) -- (0.7,0.3);
		\end{tikzpicture}}
	\quad\quad\quad\subcaptionbox{\(U_2\)}{
		\begin{tikzpicture}[scale=2]
			\begin{knot}[clip width = 5]
				\draw[thick, red] (-0.5,-0.3) -- (-0.5,0.3);
				\draw[thick, red] (-0.3,0.3) .. controls (-0.3,-0.06) and (0.3,-0.06) .. (0.3,0.3);
				\draw[thick, red] (-0.3,-0.3) .. controls (-0.3,0.06) and (0.3,0.06) .. (0.3,-0.3);
				\draw[thick, red] (0.5,-0.3) -- (0.5,0.3);
			\end{knot}
		\end{tikzpicture}}
	\quad\quad\quad\subcaptionbox{\(U_3\)}{
		\begin{tikzpicture}[scale=2]
			\begin{knot}[clip width = 5]
				\draw[thick, red] (-0.5,-0.3) -- (-0.5,0.3);
				\draw[thick, red] (-0.3,0.3) .. controls (-0.3,-0.06) and (0.3,-0.06) .. (0.3,0.3);
				\draw[thick, red] (-0.3,-0.3) .. controls (-0.3,0.06) and (0.3,0.06) .. (0.3,-0.3);
				\draw[thick, red] (-0.7,-0.3) -- (-0.7,0.3);
			\end{knot}
		\end{tikzpicture}}
	\caption{Input-output forms or hooks for 4 strands}
	\label{fig:inputoutputforms}
\end{figure}
We can thus use consider \(\sigmaa_i\) to be equivalent to \(A + A^{-1}\symup{U}_i\), \(\sigmaa^{-1}_i\) to be equivalent to \(A^{-1} + A\symup{U}_i\), and create a formalism based on this. Given a braid word representation of a braid, we can simply substitute the above mentioned equivalences to get a product of \(\symup{U}_i\)'s, with \(A\) and \(A^{-1}\) as coefficients. Note that this product is dependent on the specific braid representation of a knot. It is not invariant under a type I move, as is the case with the un-normalised bracket polynomial.

We see that each bracket polynomial evaluation state of the closure of a braid can be written in terms of the closure of a product of the input-output forms. Given a braid word \(b\), we can consider it equivalent to \(S(b)\), where \(S(b)\) is the sum of products of the \(\symup{U}_i\)'s obtained by substituting \(\sigmaa_i\) by \(A + A^{-1}\symup{U}_i\) and \(\sigmaa_i^{-1}\) by \(A^{-1} + A\symup{U}_i\). Closure of each term (a product in \(U_i\)'s) in the sum corresponds to a state obtained while evaluating the bracket polynomial as it gives a collection of loops. If \(P\) is such a product, then \(\left\langle P\,\right\rangle = \braket<\overline{P}\,> = \symup{\delta}^{\norm{P}}\), where \(\norm{P}\) is the number of loops in \(\overline{P}\) minus \(1\), and \(\symup{\delta} = -A^{-2} - A^2\). Thus, \[S(b) = \sum_s \braket<b|s> P_s,\] where \(s\) denotes a state and indexes all the terms in the product, and \(\braket<b|s>\) is the product of \(A\)'s and \(A^{-1}\)'s multiplying each \(P\)-product \(P_s\). We have \[\braket<b> = \braket<S(b)> = \sum_s \braket<b|s> \braket<P_s> = \sum_s \braket<b|s> \updelta^{\norm{s}}.\]

\begin{exmp}
	Let \(b = \sigmaa_1\sigmaa_2^{-1}\). We can resolve \(\overline{b}\) in many states. One of the states \(s\) and its corresponding product \(\U_1 \U_2\) in terms of the input-output forms is shown in \cref{fig:inputoutputexample}.
	\begin{figure}
	    \centering
		\subcaptionbox{\(b\)}{
			\begin{tikzpicture}
				\begin{knot}[clip width=5]
					\strand[very thick, blue] (2,0) to [out=90,in=-90] (1,1);
					\strand[very thick, blue] (1,1) [out=90,in=-90] to (0,2);
					\strand[very thick, blue] (1,0) to [out=90,in=-90] (2,1);
					\strand[very thick, blue] (2,1) [out=90,in=-90] to (2,2);
					\strand[very thick, blue] (0,0) to [out=90,in=-90] (0,1);
					\strand[very thick, blue] (0,1) [out=90,in=-90] to (1,2);
					\flipcrossings{2}
				\end{knot}
			\end{tikzpicture}}
		\quad\quad\subcaptionbox{\(\overline{b}\)}{
			\begin{tikzpicture}
				\begin{knot}[clip width=5]
					\strand[very thick, blue] (2,0) to [out=90,in=-90] (1,1);
					\strand[very thick, blue] (1,1) [out=90,in=-90] to (0,2);
					\strand[very thick, blue] (1,0) to [out=90,in=-90] (2,1);
					\strand[very thick, blue] (2,1) [out=90,in=-90] to (2,2);
					\strand[very thick, blue] (0,0) to [out=90,in=-90] (0,1);
					\strand[very thick, blue] (0,1) [out=90,in=-90] to (1,2);
					\strand[thick, red] (2,2) to (2,2.5) to (2.5, 2.5) to (2.5,-0.5) to (2,-0.5) to (2,0);
					\strand[thick, red] (1,2) to (1,3) to (3,3) to (3,-1) to (1,-1) to (1,0);
					\strand[thick, red] (0,2) to (0,3.5) to (3.5,3.5) to (3.5,-1.5) to (0,-1.5) to (0,0);
					\flipcrossings{2}
				\end{knot}
			\end{tikzpicture}}\par\bigskip
		\subcaptionbox{\(s = \overline{U_1 U_2}\)}{
			\begin{tikzpicture}
				\begin{knot}[clip width=5]
					\draw[very thick, blue] (1,0) .. controls (1,0.5) and (2,0.5) .. (2,0);
					\draw[very thick, blue] (1,1) .. controls (1,0.5) and (2,0.5) .. (2,1) to (2,2);
					\draw[very thick, blue] (0,0) to (0,1) .. controls (0,1.5) and (1,1.5) .. (1,1);
					\draw[very thick, blue] (0,2) .. controls (0,1.5) and (1,1.5) .. (1,2);
					\strand[thick, red] (2,2) to (2,2.5) to (2.5, 2.5) to (2.5,-0.5) to (2,-0.5) to (2,0);
					\strand[thick, red] (1,2) to (1,3) to (3,3) to (3,-1) to (1,-1) to (1,0);
					\strand[thick, red] (0,2) to (0,3.5) to (3.5,3.5) to (3.5,-1.5) to (0,-1.5) to (0,0);
				\end{knot}
			\end{tikzpicture}}
		\quad\quad\quad\subcaptionbox{\(U_1 U_2\)}{
			\begin{tikzpicture}
				\begin{knot}[clip width=5]
					\draw[very thick, blue] (1,0) .. controls (1,0.5) and (2,0.5) .. (2,0);
					\draw[very thick, blue] (1,1) .. controls (1,0.5) and (2,0.5) .. (2,1) to (2,2);
					\draw[very thick, blue] (0,0) to (0,1) .. controls (0,1.5) and (1,1.5) .. (1,1);
					\draw[very thick, blue] (0,2) .. controls (0,1.5) and (1,1.5) .. (1,2);
				\end{knot}
			\end{tikzpicture}}
		\caption{Writing a state of a braid closure in terms of input-output forms.}
		\label{fig:inputoutputexample}
	\end{figure}
\end{exmp}

\begin{exmp}
	Consider the same braid \(b = \sigmaa_1\sigmaa^{-1}_2\). We have
	\begin{align*}
	    P(b) &= (A + A^{-1}\U_1) (A\U_2 + A^{-1})\\
		P(b) &= A^2\U_2 + \I_3 + \U_1\U_2 + A^{-2}\U_1\\
		\braket<b> = \braket<\sigmaa_1\sigmaa^{-1}_2> = \braket<P(b)> &= A^2\braket<\U_2> + \braket<\I_3> + \braket<\U_1\U_2> + A^{-2}\braket<\U_1>.
	\end{align*}
	Now, \(\I_3\) corresponds to \BPIthree. Thus, \(\braket<I_3> = \updelta^{3-1} = \updelta^2\) as the closure of \(\I_3\) shall give three loops. \(\U_1\) corresponds to \KPDonethree\,. Thus, \(\braket<\U_1> = \updelta^{2-1} = \updelta\) as the closure shall give two loops. Similarly, \(\braket<\U_2> = \updelta\) and \(\braket<\U_1\U_2> = \updelta^{1-1} = 1.\) For a visual representation of the state \(\U_1\U_2\), see \cref{fig:inputoutputexample}, where the closure gives one single loop. In the end, we get \[\braket<\overline{b}> = A^2\updelta + \updelta^2 + 1 + A^{-2}\updelta.\]
\end{exmp}

\section{Temperley--Lieb algebra}

We can give the \(\U_i\)'s a structure of their own by constructing the free additive algebra \(\TL_n\) with the generators \(\U_1, \U_2,\ldots, \U_{n-1}\) and the multiplicative relations coming from the interpretation of \(\U_i\)'s as input-output forms. We can consider this algebra over the ring \(\Z[A, A^{-1}]\) with \(\updelta = -A^{-2} - A^2 \in \Z[A, A^{-1}]\). We shall call \(\TL_n\) the Temperley--Lieb algebra. The multiplicative relations in \(\TL_n\) are as follows.
\begin{enumerate}
	\item \(\U_i\U_{i\pm 1}\U_i = \U_i\).
	\item \(\U_i^2 = \updelta \U_i\).
	\item \(\U_i\U_j = \U_j\U_i\) if \(\abs{i-j} \geq 2\).
\end{enumerate}
These relations are a result of the geometric relations as illustrated in \cref{fig:inputoutputillus}.
\begin{figure}
    \centering
	\subcaptionbox{\(\U_1\U_2\U_1 = \U_1\)}{
		\begin{tikzpicture}
			\begin{knot}[clip width=5]
				\draw[thick, red] (0,0) .. controls (0,0.5) and (1,0.5) .. (1,0);
				\draw[thick, red] (2,0) to (2,1) .. controls (2,1.5) and (1,1.5) .. (1,1) .. controls (1,0.5) and (0,0.5) .. (0,1) to (0,2) .. controls (0,2.5) and (1,2.5) .. (1,2) .. controls (1,1.5) and (2,1.5) .. (2,2) to (2,3);
				\draw[thick, red] (0,3) .. controls (0,2.5) and (1,2.5) .. (1,3);
				\node[label={\(\cong\)}] at (3,1.1) {};
				\draw[thick, red] (4,0) -- (4,1) .. controls (4,1.5) and (5,1.5) .. (5,1) -- (5,0);
				\draw[thick, red] (4,3) -- (4,2) .. controls (4,1.5) and (5,1.5) .. (5,2) -- (5,3);
				\draw[thick, red] (6,0) -- (6,3);
			\end{knot}
		\end{tikzpicture}}\par\bigskip
	\subcaptionbox{\(\U_1^2 = \updelta\U_1\), we have a union of a loop(\(\updelta\)), and \(\U_1\) on the right side}{
		\begin{tikzpicture}
			\draw[thick, red] (0,2) .. controls (0,1.5) and (1,1.5) .. (1,2);
			\draw[thick, red] (0,1) .. controls (0,1.5) and (1,1.5) .. (1,1);
			\draw[thick, red] (0,0) .. controls (0,0.5) and (1,0.5) .. (1,0);
			\draw[thick, red] (0,1) .. controls (0,0.5) and (1,0.5) .. (1,1);
			\node[label={\(\cong\)}] at (2,0.7) {};
			\draw[thick, red] (3,1) .. controls (3,1.5) and (4,1.5) .. (4,1);
			\draw[thick, red] (3,1) .. controls (3,0.5) and (4,0.5) .. (4,1);
			\draw[thick, red] (4.5,0) to (4.5,0.5) .. controls (4.5,1) and (5.5,1) .. (5.5,0.5) -- (5.5,0);
			\draw[thick, red] (4.5,2) to (4.5,1.5) .. controls (4.5,1) and (5.5,1) .. (5.5,1.5) -- (5.5,2);
		\end{tikzpicture}}\par\bigskip
	\subcaptionbox{\(\U_3\U_1 = \U_1\U_3\)}{
		\begin{tikzpicture}
			\draw[thick, red] (0,0) .. controls (0,0.5) and (1,0.5) .. (1,0);
			\draw[thick, red] (0,2) -- (0,1) .. controls (0,0.5) and (1,0.5) .. (1,1) -- (1,2);
			\draw[thick, red] (2,0) -- (2,1) .. controls (2,1.5) and (3,1.5) .. (3,1) -- (3,0);
			\draw[thick, red] (2,2) .. controls (2,1.5) and (3,1.5) .. (3,2);
			\node[label={\(\cong\)}] at (4,1.1) {};
			\draw[rotate around={180:(5.5,1)}, xshift=5cm, thick, red] (0,0) .. controls (0,0.5) and (1,0.5) .. (1,0);
			\draw[rotate around={180:(5.5,1)}, xshift=5cm, thick, red] (0,2) -- (0,1) .. controls (0,0.5) and (1,0.5) .. (1,1) -- (1,2);
			\draw[rotate around={180:(7.5,1)}, xshift=5cm, thick, red] (2,0) -- (2,1) .. controls (2,1.5) and (3,1.5) .. (3,1) -- (3,0);
			\draw[rotate around={180:(7.5,1)}, xshift=5cm, thick, red] (2,2) .. controls (2,1.5) and (3,1.5) .. (3,2);
		\end{tikzpicture}}
	\caption{Input-output form relations}
	\label{fig:inputoutputillus}
\end{figure}

Now that we have the Temperley--Lieb algebra, we can define a mapping \[\uprho \colon \B_n \rightarrow \TL_n\] by the following formulae.
\begin{align*}
	\uprho (\sigmaa_i) &= A + A^{-1}\U_i,\\
	\uprho (\sigmaa_i^{-1}) &= A^{-1} + A\U_i.
\end{align*}
We see that \(\uprho\) is indeed a representation by verifying that \(\uprhoo(\sigmaa_i)\uprhoo(\sigmaa_i^{-1}) = 1\), \(\uprhoo(\sigmaa_i \sigmaa_{i+1} \sigmaa_i) = \uprhoo(\sigmaa_{i+1} \sigmaa_i \sigmaa_{i+1})\) and if  \(\abs{i-j} \geq 2\), \(\uprhoo(\sigmaa_i\sigmaa_j) = \uprhoo(\sigmaa_j\sigmaa_i)\).
\begin{prop}
	\(\uprhoo \colon \B_n \rightarrow \TL_n\) is a representation of the Artin braid group.
\end{prop}
\begin{proof}
	We first prove that \(\uprhoo(\sigmaa_i)\uprhoo(\sigmaa_i^{-1}) = 1\).
	\begin{align*}
		\uprhoo(\sigmaa_i)\uprhoo(\sigmaa_i^{-1}) &= (A + A^{-1}\U_i)(A^{-1} + A\U_i)\\
		&= 1 + (A^{-2} + A^2)\U_i + \U_i^2\\
	\end{align*}
	But since \(\U_i^2 = \updelta \U_i\) and \(\updelta = -A^{-2} - A^{2}\), we have
	\begin{align*}
		\uprhoo(\sigmaa_i)\uprhoo(\sigmaa_i^{-1}) &= 1 + (A^{-2} + A^2)\U_i + \updelta\U_i\\
		&= 1 + (A^{-2} + A^2) \U_i + (-A^{-2} - A^{2})\U_i\\
		&= 1.
	\end{align*}
	We now prove that \(\uprhoo(\sigmaa_i \sigmaa_{i+1} \sigmaa_i) = \uprhoo(\sigmaa_{i+1} \sigmaa_i \sigmaa_{i+1})\).
	\begin{align*}
		\uprhoo(\sigmaa_i \sigmaa_{i+1} \sigmaa_i) &= (A + A^{-1}\U_i)(A + A^{-1}\U_{i+1})(A + A^{-1}\U_i)\\
		&= (A^2 + \U_{i+1} + \U_i + A^{-2}\U_i\U_{i+1})(A + A^{-1}\U_i)\\
		&= A^3 + A\U_{i+1} + A\U_i + A^{-1}\U_i\U_{i+1} + A^{-2}\U_i^2 + A\U_i\\
		&\phantom{=} + A^{-1}\U_{i+1}\U_i + A^{-3}\U_i\U_{i+1}\U_i\\
		&= A^3 + A\U_{i+1} + (A^{-1}\updelta + 2A)\U_i\\
		&\phantom{=} + A^{-1}(\U_i\U_{i+1} + \U_{i+1}\U_i) + A^{-3}\U_i\\
		&= A^3 + A\U_{i+1} + (A^{-1}(-A^2 - A^{-2}) + 2A + A^{-3})\U_i\\
		&\phantom{=}+ A^{-1}(\U_i\U_{i+1} + \U_{i+1}\U_i)\\
		&= A^3 + A(\U_{i+1} + \U_i) + A^{-1}(\U_i\U_{i+1} + \U_{i+1}\U_i).
	\end{align*}
	Since symmetry of the above expression in \(i\) and \(i+1\), we can conclude that \(\uprhoo(\sigmaa_i \sigmaa_{i+1} \sigmaa_i) = \uprhoo(\sigmaa_{i+1} \sigmaa_i \sigmaa_{i+1})\).
	We now prove that if  \(\abs{i-j} \geq 2\), then \(\uprhoo(\sigmaa_i\sigmaa_j) = \uprhoo(\sigmaa_j\sigmaa_i)\).
	\begin{align*}
		\uprhoo(\sigmaa_i\sigmaa_j) &= \uprhoo(\sigmaa_i)(\sigmaa_j)\\
		&= (A + A^{-1}\U_i)(A + A^{-1}\U_j)
	\end{align*}
	Now since \(\U_i\U_j = \U_j\U_i\) if \(\abs{i-j} \geq 2\), we have \((A + A^{-1}\U_i)(A + A^{-1}\U_j) = (A + A^{-1}\U_j)(A + A^{-1}\U_i)\) which equals \(\uprhoo(\sigmaa_j\sigmaa_i)\).
\end{proof}

We now define the diagrammatic trace \(\tr \colon \TL_n \rightarrow \Z[A, A^{-1}]\) by extending linearly \(\tr(P) = \braket<P\,>\), where \(P\) is a product term in \(S(b)\). This version of trace is diagrammatic in nature as we are counting loops in a state. We thus arrive at the formula \(\braket<b> = \tr(\operatorname{\uprho}(b))\). With what we have learnt so far, one can now find a braid representation \(b\) of a link \(L\) by Alexander's theorem, calculate \(\tr(\operatorname{\uprho}(b))\) and normalise it to get the link invariant normalised bracket polynomial. One can substitute \(A = t^{-1/4}\) to arrive at our long sought destination of the Jones polynomial.

\section{Jones algebra}

Jones considered a sequence of algebras \(A_n\) for \(n = 2,3,\ldots\) with multiplicative generators \(e_1, e_2,\ldots, e_{n-1}\) and the following relations.
\begin{enumerate}
    \item \(e_i^2 = e_i\).
	\item \(e_ie_{i\pm 1} e_i = ce_i\).
	\item \(e_ie_j = e_je_i\) if \(\abs{i-j} \geq 2\).
\end{enumerate}
Here, \(c\) is a scalar which commutes with all the other elements. We can consider \(A_n\) as the free additive algebra on these generators viewed as a module over the ring \(\symbb{C}[c,c^{-1}]\), i.e.\@ we take the free ring in the generators \(e_1\), \(e_2\),\ldots, \(e_{n-1}\) and quotient it with the smallest ideal containing the elements \(e_i\), \(c^{-1}e_i e_{i+1}\) and \(e_i e_j e_i^{-1} e_j^{-1}\) with appropriately restricted \(i\) and \(j\). While \(c\) is often taken to be a complex number, we can view it as another algebraic variable which commutes with the generators. \(A_n\) arose in the theory of classification of von Neumann algebras and one can realize \(A_n\) as a von Neumann algebra as well. The reader is requested to compare the above mentioned relations for Jones algebra with the relations for the Artin braid group and the Temperley--Lieb algebra.

As mentioned in the beginning of this chapter, it is natural to construct a nested tower of algebras \[M_0 \hookrightarrow M_1 \hookrightarrow M_2 \hookrightarrow M_3\hookrightarrow \cdots \hookrightarrow M_n \hookrightarrow M_{n+1} \hookrightarrow\ldots\] with the following properties.
\begin{enumerate}
    \item \(M_0\) and \(M_1\) are given.
	\item \(e_i\colon M_i \rightarrow M_{i-1}\) is a `projection'.
	\item \(e_i^2 = e_i\).
	\item \(M_{i+1} = \left\langle M_i,e_i\right\rangle\).
\end{enumerate}
Jones constructed such a tower of algebras with the other two properties of generators \(e_ie_{i\pm 1} e_i = ce_i\) and \(e_ie_j = e_je_i\) if \(\abs{i-j} \geq 2\). He defined a notion of a trace \(\tr\colon M_n \rightarrow \C\), i.e.\@ a function which vanishes on the commutator of any two elements of \(M_n\). This trace satisfied the so called Markov property: \(\tr(we_i) = c\tr(w)\) for \(w\) in the algebra generated by \(M_0\), \(e_i\),\ldots, \(e_{i-1}\).

We now see a concrete example of such a tower of algebras. Consider \[\R \hookrightarrow \R[x_1] \hookrightarrow \R[x_1, x_2] \hookrightarrow \R[x_1, x_2, x_3] \hookrightarrow \cdots \hookrightarrow \R[x_1, x_2, \ldots, x_n] \hookrightarrow \ldots.\] We have a sequence of the set of all real numbers, the set of all real polynomials in one variable, the set of all real polynomials in two variables, the set of all real polynomials in three variables, and so on. Each variable \(x_i\) for \(i\geq 2\), is a map from \(\R[x_1, x_2, \ldots, x_{i-1}]\) to \(\R[x_1, x_2, \ldots, x_{i-2}]\), defined as the coefficient of \(x_{i-1}^0 \in \R[x_1, x_2, \ldots, x_{i-1}]\).
\begin{align*}
	x_i &\colon \R[x_1, x_2, \ldots, x_{i-1}] \rightarrow \R[x_1, x_2, \ldots, x_{i-2}]\\
	x_i &\colon \sum_{j=0}^{m}p_j x^{j}_{i-1} \mapsto p_0,
\end{align*}
where \(p_j \in \R[x_1, x_2, \ldots, x_{i-2}]\) and \(m\in\n\). As an example, if \(\sum_{j=0}^{m}p_j x^{j}_1 \in \R[x_1]\), then
\begin{align*}
    x_2 &\colon \R[x_1] \rightarrow \R\\
	x_2 &\colon \sum_{j=0}^{m}p_j x^{j}_1 \mapsto p_0,
\end{align*}
where \(p_0\) is the coefficient of \(x_1^0\), i.e.\@ the constant term. For example, \[x_2 (3x_1^4 + 6x_1^3 + x_1 + 5) = 5.\] Similarly, \(x_3\) is a map from \(\R[x_1, x_2]\) to which maps a polynomial in two variables to the polynomial (in the single variable \(x_1\)) which does not contain any power of \(x_2\). As an example, \[x_3(4x_1x_2^2 + 3x_1^3 + x_2 + x_2^5 + 1) = 3x_1^3 + 1.\]

\((\R[x_1, x_2, \ldots, x_n], +, \cdot)\) is an algebra with \(+\) defined as polynomial addition (component-wise addition) and \(\cdot\) defined as  polynomial multiplication (Cauchy product). The generator relations for the Jones algebra, however, are not satisfied with the \(\cdot\) operation; \(x_1^2 \coloneq x_1 \cdot x_1 \neq x_1\). But we can make sense of them if we interpret the operation amongst the generators as the function composition operator. \(x_1\), \(x_2\),\ldots, \(x_n\) are the generators \((\R[x_1, x_2, \ldots, x_n], +, \cdot)\). Since \(x_i\)'s are functions as well, we can interpret \(x_i^2\) as \(x_i \circ x_i\). If we do that, then \(x_i^2 \coloneq x_i \circ x_i = x_i\) as \(x_i\) is a projection operator. Similarly, we see that \(x_i \circ x_{i\pm 1}\circ x_i = x_i\) and \(x_i \circ x_j = x_j \circ x_i\) by restricting the domains appropriately.

We can define a Markov trace on the above tower as follows. For an element \(w \in \R[x_1, x_2, \ldots, x_{n}]\), define \(\tr \colon \R[x_1, x_2, \ldots, x_{n}] \rightarrow \Z\) by assigning \(w\) the degree of the polynomial. For an element of \(\R[x_1, x_2, \ldots, x_{n+1}]\), \(\tr \colon \R[x_1, x_2, \ldots, x_{n+1}] \rightarrow \Z\). Even though the domains of the trace are different in the above two examples, we can perform this abuse of notation since there is no risk of confusion here. We see that \(\tr(w x_{n+1}) = \tr(w)\). Thus, \(\tr\) is a Markov trace as it satisfies the Markov property with \(c=1\) for all \(w \in \R[x_1, x_2, \ldots, x_{n}]\) where \(n \geq 1\).

We can retrieve the Jones algebra from the Temperley--Lieb algebra by substituting \(e_i = \updelta^{-1} \U_i\), \(e_i^2 = e_i\) and \(e_ie_{i\pm 1} e_i = \updelta^{-2}e_i\). We have taken \(c = \updelta^{-2}\). Note that the underlying rings are different.




	%\chapter{Braids and the Jones polynomial}

\section{Motivation}

As remarked earlier, \todo{insert reference} Jones arrived at his polynomial indirectly while working on the theory of operator algebras. In his course of investigations, he constructed a tower of algebras nested in one another with the property that each of these algebras is generated by a set of generators satisfying a particular set of relations. A degree of similarity between these relations and the relations among the generators of the \textit{Artin braid group} was pointed out to Jones by a student during a seminar, \todo{insert reference} which led to the investigations of Jones into knot theory. Jones had defined a notion of a \textit{trace} on his algebras; more specifically a trace function obeying the \textit{Markov property}. As we shall see, one can express every link in terms of a (non-unique) \textit{braid}. Jones then defined a representation of such a braid into his algebras. The trace of an algebra representation of a braid, which is in turn obtained from the link can be calculated. The Jones polynomial was realized as such a trace. \todo{insert figure reference}

In this chapter, we shall not travel the original route of Jones to reach his polynomial as it requires the knowledge of the theory of von Neumann algebras. Instead, we shall follow the approach of Kauffman to construct a representation of the Artin braid group into the \textit{Temperley--Lieb algebra}. These algebras admit a diagrammatic intrepretation and our definition of a trace on these algebras shall be diagrammatic in nature as well. We shall eventually reach the bracket polynomial, which we already know to be equivalent to the Jones polynomial\todo{insert reference}. The Jones algebra can be recovered from the Temperley--Lieb algebra by a choice of substitutions. The Temperley--Lieb algebra arose during the study of certain statistical models in physics.\todo{insert reference} The Temperley--Lieb algebra can be viewed as a part of a broader framework of Partition algebras. \todo{insert Partition algebras wiki link} \todo{insert figure reference and construct one}

\section{Geometric representation of braids}

\subsection{Definition}

\begin{figure}[H]
	\centering
	\begin{tikzpicture}
		[scale=1.5,
		cube/.style={thick,black},
		grid/.style={very thin,gray},
		axis/.style={->,blue,thick}]

		%draw a grid in the x-y plane
		%\foreach \x in {-0.5,0,...,2.5}
		%\foreach \y in {-0.5,0,...,2.5}
		%{
		%	\draw[grid] (\x,-0.5) -- (\x,2.5);
		%	\draw[grid] (-0.5,\y) -- (2.5,\y);
		%}

		%draw the axes
		\draw[axis] (0,0,0) -- (6,0,0) node[anchor=west]{\(\symsf{x}\)};
		\draw[axis] (0,0,0) -- (0,3,0) node[anchor=west]{\(\symsf{y}\)};
		\draw[axis] (0,0,0) -- (0,0,2) node[anchor=west]{\(\symsf{z}\)};

		%draw the top and bottom of the cube
		\draw[cube] (0,0,-1) -- (0,2,-1) -- (5,2,-1) -- (5,0,-1) -- cycle;


		%draw the edges of the cube
		\draw[cube] (0,0,-1) -- (0,0,1);
		\draw[cube] (0,2,-1) -- (0,2,1);
		\draw[cube] (5,0,-1) -- (5,0,1);
		\draw[cube] (5,2,-1) -- (5,2,1);

		\begin{knot}[clip width=8]
			\strand[thick, red] (1,0,0) to [in angle=-90, out angle=90, curve through = {(1.5,1,1)}] (4,2,0) node[black, below]{\(\symsf{p_1}\)};
			\strand[thick, red] (2,0,0) to [in angle=-90, out angle=90, curve through = {(1.7,1.5,0.5)}] (1,2,0) node[black, below]{\(\symsf{p_2}\)};
			\strand[thick, red] (3,0,0) to [in angle=-90, out angle=90, curve through = {(3,0.5,0.7)}] (2,2,0) node[black, below]{\(\symsf{p_3}\)};
			\strand[thick, red] (4,0,0) to [in angle=-90, out angle=90, curve through = {(3,1.3,-0.5)}] (3,2,0) node[black, below]{\(\symsf{p_4}\)};
			\flipcrossings{2}
		\end{knot}

		\node[below, label={\(\symsf{q_1}\)}] at (1,2,0) {};
		\node[below, label={\(\symsf{q_2}\)}] at (2,2,0) {};
		\node[below, label={\(\symsf{q_3}\)}] at (3,2,0) {};
		\node[below, label={\(\symsf{q_4}\)}] at (4,2,0) {};

		\draw[cube] (0,0,+1) -- (0,2,+1) -- (5,2,+1) -- (5,0,+1) -- cycle;

	\end{tikzpicture}
	\captionof{figure}{Three dimensional geometric representation of braids.}
	\label{fig:3drepbraids}
\end{figure}

An \(n\)-braid is an element of the Artin braid group \(\B_n\), defined via the following presentation on the generators \(\sigmaa_i\), for \(1 \leq i \leq n-1 \).
\[\B_n \coloneq\Biggl\langle
\begin{array}{c|rcll}
	& \sigmaa_i \sigmaa_i^{-1} &=& \symbb{I}_n &\\
	\sigmaa_1, \ldots, \sigmaa_{n-1} & \sigmaa_i \sigmaa_{i+1} \sigmaa_i &=& \sigmaa_{i+1} \sigmaa_i \sigmaa_{i+1} &\\
	& \sigmaa_i \sigmaa_j &=& \sigmaa_j \sigmaa_i &\text{if } \abs{i-j} \geq 2
\end{array}
\Biggr\rangle.\]
We shall now see a geometric construction in the three dimensional Euclidean space to represent the Artin braid group. This shall make clear the geometric intrepretation of the relations as well.

Consider two \textit{ordered} sets of points \(L_1 \coloneq \{p_1 \coloneq (1,0,0), \ldots, p_n \coloneq (n,0,0)\}\) and \(L_2 \coloneq \{q_1 \coloneq (1,1,0), \ldots, q_n \coloneq (n,1,0)\}\) as shown in \cref{fig:3drepbraids} for \(n=4\). Elements of \(L_1\) are called bottom points and elements of \(L_2\) are called the top points. For \(1 \leq i \leq n\), consider a family of non-intersecting continuous curves \(\gamma_i \colon [0, 1] \rightarrow \rthree\) such that
\begin{enumerate}
	\item \(\gamma_i (0) = p_i\) and \(\gamma_i (1) = q_j\) for \(1 \leq i, j \leq n\).
	\item Any plane perpendicular to the \(xy\)-plane and parallel to the \(x\)-axis intersects each of the curves either exactly once or not at all.
	\item All the curves lie in the cube determined by the vertices \((0,0,1)\), \((0,0,-1)\), \((0,1,1)\), \((0,1,-1)\), \((n+1,0,1)\), \((n+1,0,-1)\), \((n+1,1,1)\), \((n+1,1,-1)\).
\end{enumerate}
Such a labelled curve is called a strand in standard position and a family of such labelled \(n\) stands is called an \(n\)-strand set in a standard position. We can ambient isotope or rigidly move an \(n\)-strand set to get another \(n\)-strand set, possibly not in a standard position. Two \(n\)-strand sets are said to be equivalent if they are related by a sequence of rigid motions of the strand sets, and ambient isotopies of the strand sets such that the space outside the cube, along with the endpoints, remains fixed. We shall refer to an equivalence class of such \(n\)-strands as a geometric \(n\)-braid. Thus, a geometric \(n\)-braid is well-defined.

\begin{remark}
	Even though we have restricted our strands to the a bounded cube in the standard position, we can in principle change the bounds of our cube in \(x\) and \(z\) directions to any value and get the same theory. We shall not pursue this approach here.
\end{remark}

\subsection{Standard projection}

We call the projection of a standard position \(n\)-strand set onto the \(xy\)-plane to be a two dimensional representation of a braid. Such a projection is drawn in \cref{fig:2drepbraids}. It should be noted that a standard position \(n\)-strand set is unique only upto ambient isotopy, thus correspondingly the two dimensional representation of such a set is also unique only upto ambient isotopy, namely the ambient isotopies of the projection of the cube and the ambient isotopies such that the projection is a two dimensional representation of a braid for all times.

\begin{figure}[H]
	\centering
	\begin{tikzpicture}
		[scale=1.5,
		cube/.style={thick,black},
		grid/.style={very thin,gray},
		axis/.style={->,blue,thick}]

		%draw a grid in the x-y plane
		%\foreach \x in {-0.5,0,...,2.5}
		%\foreach \y in {-0.5,0,...,2.5}
		%{
		%	\draw[grid] (\x,-0.5) -- (\x,2.5);
		%	\draw[grid] (-0.5,\y) -- (2.5,\y);
		%}

		%draw the axes
		\draw[axis] (0,0,0) -- (5,0) node[anchor=west]{\(\symsf{x}\)};
		\draw[axis] (0,0,0) -- (0,2.8) node[anchor=west]{\(\symsf{y}\)};

		\begin{knot}[clip width=8]
			\strand[thick, red] (1,0) to [in angle=-90, out angle=90, curve through = {(1.5,1)}] (4,2) node[black, below]{\(\symsf{p_1}\)};
			\strand[thick, red] (2,0) to [in angle=-90, out angle=90, curve through = {(1.7,1.5)}] (1,2) node[black, below]{\(\symsf{p_2}\)};
			\strand[thick, red] (3,0) to [in angle=-90, out angle=90, curve through = {(3,0.5)}] (2,2) node[black, below]{\(\symsf{p_3}\)};
			\strand[thick, red] (4,0) to [in angle=-90, out angle=90, curve through = {(3,1.3)}] (3,2) node[black, below]{\(\symsf{p_4}\)};
			\flipcrossings{2}
		\end{knot}

		\node[below, label={\(\symsf{q_1}\)}] at (1,2) {};
		\node[below, label={\(\symsf{q_2}\)}] at (2,2) {};
		\node[below, label={\(\symsf{q_3}\)}] at (3,2) {};
		\node[below, label={\(\symsf{q_4}\)}] at (4,2) {};
	\end{tikzpicture}
	\captionof{figure}{Two dimensional geometric representation of braids.}
	\label{fig:2drepbraids}
\end{figure}

Now onwards, we shall always visually represent geometric \(n\)-braids using their standard two dimensional projections.

\subsection{Group structure}

Multiplication of any two \(n\)-geometric braids \(b_1\) and \(b_2\), denoted by \(b_1 b_2\) is defined as follows (\cref{fig:braidmultiplication}). Ambient isotope and then rigidly move \(b_1\) and \(b_2\) \textit{separately} in the standard position. Now translate only \(b_1\) in the \(+y\) direction by unit distance. The bottom points of \(b_1\) and the top points of \(b_2\) now co-incide. Concatenate their strands and shrink the concatenated strands in the \(y\) direction by half keeping fixed the bottom points of \(b_2\). The result is another geometric \(n\)-braid \(b_1 b_2\) in the standard position. Multiplication defined this way is associative.

\begin{figure}[H]
	\centering
	\subcaptionbox[t]{\(\symsf{b_1}\)}{
		\begin{tikzpicture}
			[scale=1,
			cube/.style={thick,black},
			grid/.style={very thin,gray},
			axis/.style={->,blue,thick}]

			%draw a grid in the x-y plane
			%\foreach \x in {-0.5,0,...,2.5}
			%\foreach \y in {-0.5,0,...,2.5}
			%{
			%	\draw[grid] (\x,-0.5) -- (\x,2.5);
			%	\draw[grid] (-0.5,\y) -- (2.5,\y);
			%}

			%draw the axes
			\draw[axis] (0,0) -- (4,0) node[anchor=west]{\(\symsf{x}\)};
			\draw[axis] (0,0) -- (0,2.8) node[anchor=west]{\(\symsf{y}\)};

			\begin{knot}[clip width=5]
				\strand[thick, red] (1,0) to [in angle=-90, out angle=90, curve through = {(1.5,0.5)}] (1,2) node[black, below]{\(\symsf{p_1}\)};
				\strand[thick, red] (2,0) to [in angle=-90, out angle=90, curve through = {(1.9,0.5)}] (3,2) node[black, below]{\(\symsf{p_2}\)};
				\strand[thick, red] (3,0) to [in angle=-90, out angle=90, curve through = {(2,1.3)}] (2,2) node[black, below]{\(\symsf{p_3}\)};
				\flipcrossings{2}
			\end{knot}

			%\node[below, label={\(\symsf{q_1}\)}] at (1,2) {};
			%\node[below, label={\(\symsf{q_2}\)}] at (2,2) {};
			%\node[below, label={\(\symsf{q_3}\)}] at (3,2) {};

			%\node at (2, -1.1) {\(\symsf{b_1}\)};
		\end{tikzpicture}\label{fig:braidmultiplication1}}\qquad
	\subcaptionbox[t]{\(\symsf{b_2}\)}{
		\begin{tikzpicture}
			[scale=1,
			cube/.style={thick,black},
			grid/.style={very thin,gray},
			axis/.style={->,blue,thick}]

			%draw a grid in the x-y plane
			%\foreach \x in {-0.5,0,...,2.5}
			%\foreach \y in {-0.5,0,...,2.5}
			%{
			%	\draw[grid] (\x,-0.5) -- (\x,2.5);
			%	\draw[grid] (-0.5,\y) -- (2.5,\y);
			%}

			%draw the axes
			\draw[axis] (0,0) -- (4,0) node[anchor=west]{\(\symsf{x}\)};
			\draw[axis] (0,0) -- (0,2.8) node[anchor=west]{\(\symsf{y}\)};

			\begin{knot}[clip width=5]
				\strand[thick, red] (1,0) to [in angle=-90, out angle=90, curve through = {(1.7,1)}] (2,2) node[black, below]{\(\symsf{p_1'}\)};
				\strand[thick, red] (2,0) to [in angle=-90, out angle=90, curve through = {(1.7,0.5)}] (1,2) node[black, below]{\(\symsf{p_2'}\)};
				\strand[thick, red] (3,0) to [in angle=-90, out angle=90, curve through = {(3,0.5)}] (3,2) node[black, below]{\(\symsf{p_3'}\)};
			\end{knot}

			%\node[below, label={\(\symsf{q_1'}\)}] at (1,2) {};
			%\node[below, label={\(\symsf{q_2'}\)}] at (2,2) {};
			%\node[below, label={\(\symsf{q_3'}\)}] at (3,2) {};

			%\node at (2, -1.2) {\(\symsf{b_2}\)};
		\end{tikzpicture}\label{fig:braidmultiplication2}}
	\subcaptionbox[t]{\(\symsf{b_1 b_2}\)}{
		\begin{tikzpicture}
			[scale=1,
			cube/.style={thick,black},
			grid/.style={very thin,gray},
			axis/.style={->,blue,thick}]

			%draw a grid in the x-y plane
			%\foreach \x in {-0.5,0,...,2.5}
			%\foreach \y in {-0.5,0,...,2.5}
			%{
			%	\draw[grid] (\x,-0.5) -- (\x,2.5);
			%	\draw[grid] (-0.5,\y) -- (2.5,\y);
			%}

			%draw the axes
			\draw[axis] (0,0) -- (4,0) node[anchor=west]{\(\symsf{x}\)};
			\draw[axis] (0,0) -- (0,5) node[anchor=west]{\(\symsf{y}\)};

			\begin{knot}[clip width=5]
				\strand[thick, red] (1,0) to [in angle=-90, out angle=90, curve through = {(1.7,1)}] (2,2) node[black, below]{\(\symsf{p_1''}\)};
				\strand[thick, red] (2,0) to [in angle=-90, out angle=90, curve through = {(1.7,0.5)}] (1,2) node[black, below]{\(\symsf{p_2''}\)};
				\strand[thick, red] (3,0) to [in angle=-90, out angle=90, curve through = {(3,0.5)}] (3,2) node[black, below]{\(\symsf{p_3''}\)};
				\strand[thick, red] (1,2) to [in angle=-90, out angle=90, curve through = {(1.5,2.5)}] (1,4);
				\strand[thick, red] (2,2) to [in angle=-90, out angle=90, curve through = {(1.9,2.5)}] (3,4);
				\strand[thick, red] (3,2) to [in angle=-90, out angle=90, curve through = {(2,3.3)}] (2,4);
				%\strand[thick, red] (1,0) to [in angle=-90, out angle=90, curve through = {(2,1)}] (3,4) node[black, below]{\(\symsf{p_1}\)};
				%\strand[thick, red] (2,0) to [in angle=-90, out angle=90, curve through = {(1,1)}] (1,4) node[black, below]{\(\symsf{p_2}\)};
				%\strand[thick, red] (3,0) to [in angle=-90, out angle=90, curve through = {(3,1)}] (2,4) node[black, below]{\(\symsf{p_3}\)};
				%\strand[thick, red] (1,2) to [in angle=-90, curve through = {(1.7,2)}] (2,4);
				%\strand[thick, red] (2,2) to [in angle=-90, curve through = {(1.7,1.5)}] (1,4) node[black, below];
				%\strand[thick, red] (3,2) to [in angle=-90, curve through = {(3,1.5)}] (3,4) node[black, below];
			\end{knot}

			%\node[below, label={\(\symsf{q_1''}\)}] at (1,4) {};
			%\node[below, label={\(\symsf{q_2''}\)}] at (2,4) {};
			%\node[below, label={\(\symsf{q_3''}\)}] at (3,4) {};

			%\node at (2, -1.2) {\(\symsf{b_1 b_2}\)};
		\end{tikzpicture}\label{fig:braidmultiplication3}}
	\caption{Multiplication of two braids (before shrinking).}\label{fig:braidmultiplication}
\end{figure}

We shall now drop the axes as well while representing two dimensional geometric \(n\)-braids.

An \(n\)-strand set such that each \(\gamma_i\) is a straight line segment connecting the \(i^\text{th}\) bottom point to the \(i^\text{th}\) top point is called the identity geometric \(n\)-braid and is denoted by \(\In\) (\cref{fig:braididentity}). Here we have used the same identity symbol of the Artin braid group. This abuse of notation shall not matter in the long run because we shall see that our \(n\)-strand set is isomorphic to the Artin braid group.

\begin{figure}[H]\centering
	\begin{tikzpicture}
		[scale=0.7,
		cube/.style={thick,black},
		grid/.style={very thin,gray},
		axis/.style={->,blue,thick}]

		%draw a grid in the x-y plane
		%\foreach \x in {-0.5,0,...,2.5}
		%\foreach \y in {-0.5,0,...,2.5}
		%{
		%	\draw[grid] (\x,-0.5) -- (\x,2.5);
		%	\draw[grid] (-0.5,\y) -- (2.5,\y);
		%}

		%draw the axes
		\begin{knot}[clip width=5]
			\strand[thick, red] (1,0) to (1,2);
			\strand[thick, red] (2,0) to (2,2);
			\strand[thick, red] (3,0) to (3,2);
			\flipcrossings{2}
		\end{knot}

		\node[below, label={\(\symsf{q_1}\)}] at (1,2) {};
		\node[below, label={\(\symsf{q_2}\)}] at (2,2) {};
		\node[below, label={\(\symsf{q_3}\)}] at (3,2) {};
		\node[below, label={\(\symsf{p_1}\)}] at (1,-0.7) {};
		\node[below, label={\(\symsf{p_2}\)}] at (2,-0.7) {};
		\node[below, label={\(\symsf{p_3}\)}] at (3,-0.7) {};
	\end{tikzpicture}
	\captionof{figure}{The identity \(\symsf{\I_3}\)}
	\label{fig:braididentity}
\end{figure}

A geometric \(n\)-braid \(a\) such that \(ab = ba = \In\) for some geometric \(n\)-braid \(b\) is called the inverse of \(b\) and denoted is by \(b^{-1}\). We shall see that each element has an inverse.

With these operations, the set of geometric \(n\)-braids becomes a group, which we shall denote by \(\GB_n\).

\subsection{Generators}

By the virtue of ambient isotopy, we can move the crossings in a two dimensional representation of a geometric \(n\)-braid such that each crossing lies in a region bounded by two lines parallel to the \(x\)-axis. Moreover, we can arrange the crossings such that each such region contains only one crossing. Thus, if we give the information regarding the type of each crossing for each such region, we can faithfully reconstruct the two dimensional representation. To this end, we define the generators of a geometric \(n\)-braid.

Denote by \(\tauu_i\) the geometric \(n\)-braid such that
\begin{enumerate}
	\item \(\gamma_i (1) = q_{i+1}\), \(\gamma_{i+1} (1) = q_{i}\), and \(\gamma_j (1) = q_j\) when \(j\) does not equal \(i\) or \(i+1\).
	\item \(\pii_{xy}(\gamma_i(t)) \geq 0\) and \(\pii_{xy}(\gamma_{i+1}(t)) \leq 0\) for all \(t \in [0,1]\).
\end{enumerate}
\(\pii_{xy}\) is the projection maps onto to \(xy\)-plane. \(\tauu_1, \ldots, \tauu_{n-1}\) are the generators of \(\GB_n\) (\cref{fig:geometricbraidgenerators}).

\begin{figure}[H]\centering
	\subcaptionbox{\(\symsf{\tauu_i}\)}{
		\begin{tikzpicture}
			[scale=0.7,
			cube/.style={thick,black},
			grid/.style={very thin,gray},
			axis/.style={->,blue,thick}]

			%draw a grid in the x-y plane
			%\foreach \x in {-0.5,0,...,2.5}
			%\foreach \y in {-0.5,0,...,2.5}
			%{
			%	\draw[grid] (\x,-0.5) -- (\x,2.5);
			%	\draw[grid] (-0.5,\y) -- (2.5,\y);
			%}

			%draw the axes
			\begin{knot}[clip width=5]
				\strand[thick, red] (0,0) to (2,2);
				\strand[thick, red] (2,0) to (0,2);
			\end{knot}

			\node[below, label={\(\symsf{q_i}\)}] at (0,2) {};
			\node[below, label={\(\symsf{q_{i+1}}\)}] at (2,2) {};
			\node[below, label={\(\symsf{p_i}\)}] at (0,-0.7) {};
			\node[below, label={\(\symsf{p_{i+1}}\)}] at (2,-0.7) {};

			%\node at (1, -1.1) {\(\symsf{\tauu_i}\)};
		\end{tikzpicture}}
	\qquad\quad\subcaptionbox{\(\symsf{\tauu_i^{-1}}\)}{
		\begin{tikzpicture}
			[scale=0.7,
			cube/.style={thick,black},
			grid/.style={very thin,gray},
			axis/.style={->,blue,thick}]

			%draw a grid in the x-y plane
			%\foreach \x in {-0.5,0,...,2.5}
			%\foreach \y in {-0.5,0,...,2.5}
			%{
			%	\draw[grid] (\x,-0.5) -- (\x,2.5);
			%	\draw[grid] (-0.5,\y) -- (2.5,\y);
			%}

			%draw the axes
			\begin{knot}[clip width=5]
				\strand[thick, red] (0,0) to (2,2);
				\strand[thick, red] (2,0) to (0,2);
				\flipcrossings{1}
			\end{knot}

			\node[below, label={\(\symsf{q_i}\)}] at (0,2) {};
			\node[below, label={\(\symsf{q_{i+1}}\)}] at (2,2) {};
			\node[below, label={\(\symsf{p_i}\)}] at (0,-0.7) {};
			\node[below, label={\(\symsf{p_{i+1}}\)}] at (2,-0.7) {};

			%\node at (1, -1.1) {\(\symsf{\tauu_i^{-1}}\)};
		\end{tikzpicture}}
	\caption{Generators \(\symsf{\tauu_i}\) and \(\symsf{\tauu_i^{-1}}\). We have ommited the other straight strands.}
	\label{fig:geometricbraidgenerators}
\end{figure}

\noindent For example, in \cref{fig:braidmultiplication} we have \(b_1 = \tauu_2 \in \B_3\), \(b_2 = \tauu_1 \in \B_3\) and \(b_1 b_2 = \tauu_2 \tauu_1 \in \B_3\).

If we multiply \(\tauu_i\) and \(\tauu_i^{-1}\) to form \(\tauu_i \tauu_i^{-1}\), we observe that \(\tauu_i \tauu_i^{-1} = \I_n\), where \(\tauu_i, \tauu_i^{-1} \in \B_n\) for all \(n \geq 2\) (\cref{fig:type2}).

\begin{figure}[H]\centering
	\begin{tikzpicture}
		[scale=0.7]

		\begin{knot}[clip width=5]
			\strand[thick, red] (0,0) to [in angle=-90, out angle=90, curve through = {(2,2)}] (0,4);
			\strand[thick, red] (2,0) to [in angle=-90, out angle=90, curve through = {(0,2)}] (2,4);
			\flipcrossings{1,2}
		\end{knot}

		\node at (1, -0.7) {\(\symsf{\tauu_i \tauu_i^{-1}}\)};
	\end{tikzpicture}
	\quad\quad\quad\quad
	\begin{tikzpicture}
		[scale=0.7]

		\begin{knot}[clip width=5]
			\strand[thick, red] (0,0) to (0,4);
			\strand[thick, red] (2,0) to (2,4);
			\flipcrossings{1,2}
		\end{knot}

		\node at (1, -0.7) {\(\symsf{\I_n}\)};
	\end{tikzpicture}

	\captionof{figure}{A type II move illustrating \(\symsf{\tauu_i \tauu_i^{-1} = \I_n}\)}
	\label{fig:type2}
\end{figure}

We can perform a move equivalent to the type III move to see that \(\tauu_i \tauu_{i+1} \tauu_i = \tauu_{i+1} \tauu_i \tauu_{i+1}\) (\cref{fig:type3}).

\begin{figure}[H]\centering
	\begin{tikzpicture}
		[scale=0.6]

		\begin{knot}[clip width=5]
			\strand[ultra thick, blue] (0,0) to [in angle=-90, out angle=90, curve through = {(2,2) (4,4)}] (4,6);
			\strand[thick, red] (2,0) to [in angle=-90, out angle=90, curve through = {(0,2) (0,4)}] (2,6);
			\strand[thick, red] (4,0) to [in angle=-90, out angle=90, curve through = {(4,2) (2,4)}] (0,6);
		\end{knot}

		\node at (2, -0.7) {\(\symsf{\tauu_i \tauu_{i+1} \tauu_i}\)};
	\end{tikzpicture}
	\quad\quad\quad\quad
	\begin{tikzpicture}
		[scale=0.6]

		\begin{knot}[clip width=5]
			\strand[ultra thick, blue] (0,0) to [in angle=-90, out angle=90, curve through = {(0,2) (2,4)}] (4,6);
			\strand[thick, red] (2,0) to [in angle=-90, out angle=90, curve through = {(4,2) (2,4)}] (2,6);
			\strand[thick, red] (4,0) to [in angle=-90, out angle=90, curve through = {(2,2) (0,4)}] (0,6);
			%\flipcrossings{1,2}
		\end{knot}

		\node at (2, -0.7) {\(\symsf{\tauu_{i+1} \tauu_i \tauu_{i+1}}\)};
	\end{tikzpicture}

	\captionof{figure}{A type III move illustrating \(\symsf{\tauu_i \tauu_{i+1} \tauu_i = \tauu_{i+1} \tauu_i \tauu_{i+1}}\).}
	\label{fig:type3}
\end{figure}

We can slide two crossings vertically across each other if this does not change the ambient isotopy type. This is possible if the two crossings we wish to slide do not share a strand. This gives us the relation \(\tauu_i \tauu_j = \tauu_j \tauu_i\) if \(\abs{i - j} \geq 2\) (\cref{fig:sliding}).

\begin{figure}[H]\centering
	\begin{tikzpicture}
		[scale=0.5]

		\begin{knot}[clip width=5]
			\strand[thick, red] (0,0) to [in angle=-90, out angle=90, curve through = {(2,2)}] (2,6);
			\strand[thick, red] (2,0) to [in angle=-90, out angle=90, curve through = {(0,2)}] (0,6);
			\strand[thick, red] (4,0) to [in angle=-90, out angle=90, curve through = {(4,4)}] (6,6);
			\strand[thick, red] (6,0) to [in angle=-90, out angle=90, curve through = {(6,4)}] (4,6);
		\end{knot}

		\node at (3, -0.7) {\(\symsf{\tauu_i \tauu_j}\)};
	\end{tikzpicture}
	\quad\quad\quad\quad
	\begin{tikzpicture}
		[scale=0.5]

		\begin{knot}[clip width=5]
			\strand[thick, red] (0,0) to [in angle=-90, out angle=90, curve through = {(0,4)}] (2,6);
			\strand[thick, red] (2,0) to [in angle=-90, out angle=90, curve through = {(2,4)}] (0,6);
			\strand[thick, red] (4,0) to [in angle=-90, out angle=90, curve through = {(6,2)}] (6,6);
			\strand[thick, red] (6,0) to [in angle=-90, out angle=90, curve through = {(4,2)}] (4,6);
			%\flipcrossings{1,2}
		\end{knot}

		\node at (3, -0.7) {\(\symsf{\tauu_j \tauu_i}\)};
	\end{tikzpicture}

	\captionof{figure}{Sliding of crossings illustrating \(\symsf{\tauu_i \tauu_j = \tauu_j \tauu_i}\).}
	\label{fig:sliding}
\end{figure}

Thus, we can define \(\GB_n\) as the quotient of the free group \(F\) generated by \(\{\tauu_1, \ldots, \tauu_{n-1}\}\) with the smallest normal subgroup containing the elements \(\tauu_i \tauu_i^{-1}\), \(\tauu_i \tauu_{i+1} \tauu_i \tauu_{i+1}^{-1} \tauu_i^{-1} \tauu_{i+1}^{-1}\), and if \(\abs{i-j} \geq 2\), then \(\tauu_i \tauu_j \tauu_i^{-1} \tauu_{j}^{-1}\).
\[\GB_n \coloneq\Biggl\langle
\begin{array}{c|rcll}
	& \tauu_i \tauu_i^{-1} &=& \symbb{I}_n &\\
	\tauu_1, \ldots, \tauu_{n-1} & \tauu_i \tauu_{i+1} \tauu_i &=& \tauu_{i+1} \tauu_i \tauu_{i+1} &\\
	& \tauu_i \tauu_j &=& \tauu_j \tauu_i &\text{if } \abs{i-j} \geq 2
\end{array}
\Biggr\rangle.\]

We thus define a map
\begin{align*}
	\upphi \colon &\{\sigmaa_1, \ldots, \sigmaa_{n-1}\} \rightarrow \{\tauu_1, \ldots, \tauu_{n-1}\},\\
	\upphi \colon &\sigmaa_i \mapsto \tauu_i.
\end{align*}
Let \(w\) be a word of length \(m\) in \(\B_n\); \(w = \prod_{j=1}^{m} \sigmaa^{\pm 1}_{\alpha_j}\) where \(1 \leq \alpha_j \leq n-1\). We can define a group isomorphism
\begin{align*}
	\upPhi &\colon \B_n \rightarrow \GB_n,\\
	\upPhi &\colon \prod_{j=1}^{m} \sigmaa^{\pm 1}_{\alpha_j} \mapsto \prod_{j=1}^{m} \upphi(\sigmaa^{\pm 1}_{\alpha_j}) \text{ for all } m \in \symbb{N}.
\end{align*}
We have finally succeeded in retrieving the algebraically defined Artin braid group from the geometric definition.

\section{Closure of braids}

\subsection{Introduction}

We define the closure of a geometric \(n\)-braid as follows. Consider a geometric \(n\)-braid in the standard position. For each \(1 \leq i \leq n\), we construct the following sequence of line connected line segments. Join \((i, 1, 0)\), \((i, i, 0)\), \((i, i, 0)\), \((i, -i, 0)\), \((i, -i, 0)\), \((i, 0, 0)\) consecutively. We then join \(\gamma_i\) to the constructed line segments. Repeating this process for all \(i\) gives the closure of a braid (\cref{fig:closure}). We denote the closure of a geometric \(n\)-braid \(b\) by \(\overline{b}\). Closure of a braid is unique upto ambient isotopy.

\begin{figure}[H]\centering
	\subcaptionbox{\(\symsf{b}\)}{
		\begin{tikzpicture}
			[scale=0.6]

			\begin{knot}[clip width=5]
				\strand[thick, red] (0,0) to[in angle=-90, out angle=90, curve through = {(1,1)}] (2,2);
				\strand[thick, red] (2,0) to[in angle=-90, out angle=90, curve through = {(1,1)}] (0,2);
				\strand[thick, red] (4,0) to (4,2);
				%\strand[thick, red] (6,0) to [in angle=-90, out angle=90, curve through = {(6,4)}] (4,6);
			\end{knot}

			%\node at (2, -0.7) {\(\symsf{b}\)};
		\end{tikzpicture}}
	\quad\quad\subcaptionbox{\(\symsf{\overline{b}}\)}{
		\begin{tikzpicture}
			[scale=0.6]

			\begin{knot}[clip width=5]
				\strand[thick, blue] (0,0) to[in angle=-90, out angle=90, curve through = {(1,1)}] (2,2);
				\strand[thick, blue] (2,0) to[in angle=-90, out angle=90, curve through = {(1,1)}] (0,2);
				\strand[thick, blue] (4,0) to (4,2);
				\strand[thick, red] (4,2) to (4,3) to (6,3) to (6,-1) to (4,-1) to (4,0);
				\strand[thick, red] (2,2) to (2,4) to (8,4) to (8,-2) to (2,-2) to (2,0);
				\strand[thick, red] (0,2) to (0,5) to (10,5) to (10,-3) to (0,-3) to (0,0);
				%\strand[thick, red] (0,4) t
				%\flipcrossings{1,2}
			\end{knot}

			%\node at (5, -6.7) {\(\symsf{\overline{b}}\)};
		\end{tikzpicture}}
	\caption{Closure of a braid with \(\symsf{b = \tauu_1 \in \GB_2}\).}
	\label{fig:closure}
\end{figure}

\begin{prop}
	Every closure of a geometric \(n\)-braid is a link.
\end{prop}
\begin{proof}[Proof]
	The outer line segments are locally flat by virtue of being piecewise linear. Due to the second condition regarding the intersection of a plane in the standard representation of geometric \(n\)-braid, we can project the strand onto the \(y\)-axis and this projection would be a homeomorphism. We take an \(\epsilon\) neighbourhood, \(N_\epsilon\) around the strand. The pair \((N_\epsilon, \gamma_i)\) would then be homeomorphic to \((\symup{B}, \symup{B} \cap x\text{-axis})\), where \(\symup{B}\) is the three-dimensional unit ball around the origin. We have local flatness at the end points as well due to the union of two locally flat curves.
\end{proof}

\begin{remark}
	Suppose \(\tauu_i \in \GB_n\) and \(\tauu_i' \in \GB_{m}\) are generators where \(n < m\). Then the closures of these two generators are not ambient isotopic. The closure of the latter contains one more non-linking loop.
\end{remark}

\subsection{Alexander's theorem}

\begin{thm}[Alexander]
	Every link is ambient isotopic to the closure of a geometric braid, for some \(n \in \symbb{N}\).
\end{thm}
\begin{proof}
	Consider a piecewise linear, regular projection \(\pi(L)\) of a link \(L\) on a plane. We choose a point \(O\) in the projection plane which is not collinear with any of the line segments. This can be done since a the link has only finitely many line segments. Let \(P \in \pi(L)\). The vector \(OP\) can move either clockwise or anti-clockwise as \(P\) moves along the link projection. We wish to modify the line segments such that \(OP\) moves in only one sense, say anti-clockwise, as \(P\) moves along the entire length of the link projection. We now fix our attention on a line segment corresponding to a clockwise rotation. We divide the segment into sub-parts such that each part shares at-most one crossing point with other line segments. If \(A\) and \(B\) are end-points of such a line segment, then we may replace this line segment with two another line segments \(AC\) and \(CB\), such that \(C\) is another point not on belonging to \(\pi(L)\) and the triangle \(ABC\) encloses \(O\). If \(AB\) originally passed under (or over) a line segment of \(\pi(L)\), then the modified line segments \(AC\) and \(CB\) must pass under (or over) of the line segments of \(\pi(L)\) as well. This move shall not change the link type as it shall be a combination of sliding, type 2 and type 3 moves. In the resulting triangle, the vector \(OP\) shall move in the opposite, anti-clockwise sense while traversing from \(A\) to \(B\) via \(C\), instead of \(AB\). We can repeat this process for all of the (finitely many) line segments which turn clockwise. In the end, we obtain a projection such that \(OP\) moves in only the anti-clockwise sense, as \(P\) moves along the entire length of the link projection. We can ambient isotope the projection such that all the crossings lie in the projection of a cube, more precisely the cube constructed while defining a geometric braid. The end-points can be made to match as well. The above procedure of triangluar moves shall guarantee the monotonicity that is required.
\end{proof}

Henceforth, unless specified otherwise, we shall always work with the projections of the standard representation of a geometric \(n\)-braid and its closure and refer to these projections simply as a braid and its closure.

\subsection{Conjugation}

If \(b, g \in \GB_n\), we observe that \(\overline{g b g^{-1}}\) is ambient isotopic to \(\overline{b}\) (\cref{fig:conjugation}). The upper strands in \(g\) and \(g^{-1}\) are connected via the closure strand. We can slide \(g\) and \(g^{-1}\) via the closure strands to the `other side' of \(b\) to annihilate each other. This can be achieved by a type II move.

\begin{figure}[H]
	\centering
	\subcaptionbox{\(\symsf{\overline{gbg^{-1}}}\)}{
		\begin{tikzpicture}
			\begin{knot}[clip width = 5]
				\strand[thick, red] (0,0) to [in=-90, out=90](1,1);
				\strand[thick, red] (0,2) to [out=90, in=-90](1,3) to (1,3.5) to (1.5,3.5) to (1.5,-0.5) to (1,-0.5) to (1,0);
				\strand[thick, red] (1,0) to [in=-90, out=90](0,1);
				\strand[thick, red] (1,2) to [out=90, in=-90](0,3) to (0,4) to (2,4) to (2,-1) to (0,-1) to (0,0);
				\strand[very thick, blue] (1,1) to [out=90,in=-90, blue] (0,2);
				\strand[very thick, blue] (0,1) to [out=90,in=-90, blue] (1,2);
				\flipcrossings{1,2}
			\end{knot}
			\node at (-0.5,0.5) {\(\symsf{g^{-1}}\)};
			\node at (-0.5,1.5) {\(\symsf{b}\)};
			\node at (-0.5,2.5) {\(\symsf{g}\)};
		\end{tikzpicture}}
	\quad\quad\subcaptionbox{}{
		\begin{tikzpicture}
			\begin{knot}[clip width = 5]
				\strand[thick, red] (0,2) to (0,3.5) to (1.5,3.5) to (2.5,2.5) to (3,2.5) to (3,0.5) to (2.5,0.5) to (1.5,-0.5) to (0,-0.5) to (0,1);
				\strand[thick, red] (1,2) to (1,2.5) to (1.5,2.5) to (2.5,3.5) to (3.5,3.5) to (3.5,-0.5) to (2.5,-0.5) to (1.5,0.5) to (1,0.5) to (1,1);
				\strand[very thick, blue] (1,1) to [out=90,in=-90, blue] (0,2);
				\strand[very thick, blue] (0,1) to [out=90,in=-90, blue] (1,2);
				\flipcrossings{1,2}
			\end{knot}
			\node at (2,-1) {\(\symsf{g^{-1}}\)};
			\node at (-0.5,1.5) {\(\symsf{b}\)};
			\node at (2,4) {\(\symsf{g}\)};
		\end{tikzpicture}}
	\quad\quad\subcaptionbox{}{
		\begin{tikzpicture}
			\begin{knot}[clip width = 5]
				\strand[thick, red] (0,2) to (0,3.5) to (1.5,3.5) to (2.5,2.5) to (3,2.5) to (3,0.5) to (2.5,0.5) to (1.5,-0.5) to (0,-0.5) to (0,1);
				\strand[thick, red] (1,2) to (1,2.5) to (1.5,2.5) to (2.5,3.5) to (3.5,3.5) to (3.5,-0.5) to (2.5,-0.5) to (1.5,0.5) to (1,0.5) to (1,1);
				\strand[very thick, blue] (1,1) to [out=90,in=-90, blue] (0,2);
				\strand[very thick, blue] (0,1) to [out=90,in=-90, blue] (1,2);
				\flipcrossings{1,2}
			\end{knot}
			\node at (2,-1) {\(\symsf{g^{-1}}\)};
			\node at (-0.5,1.5) {\(\symsf{b}\)};
			\node at (2,4) {\(\symsf{g}\)};
		\end{tikzpicture}}
	\quad\quad\subcaptionbox{\(\symsf{\overline{b}}\)}{
		\begin{tikzpicture}
			\begin{knot}[clip width = 5]
				\strand[thick, red] (0,2) to (0,3.5) to (1.5,3.5) to (2.5,2.5) to (3,2.5) to (3,0.5) to (2.5,0.5) to (1.5,-0.5) to (0,-0.5) to (0,1);
				\strand[thick, red] (1,2) to (1,2.5) to (1.5,2.5) to (2.5,3.5) to (3.5,3.5) to (3.5,-0.5) to (2.5,-0.5) to (1.5,0.5) to (1,0.5) to (1,1);
				\strand[very thick, blue] (1,1) to [out=90,in=-90, blue] (0,2);
				\strand[very thick, blue] (0,1) to [out=90,in=-90, blue] (1,2);
				\flipcrossings{1,2}
			\end{knot}
			\node at (2,-1) {\(\symsf{g^{-1}}\)};
			\node at (-0.5,1.5) {\(\symsf{b}\)};
			\node at (2,4) {\(\symsf{g}\)};
		\end{tikzpicture}}
	\caption{Conjugation process illustrating the link equivalence of \(\symsf{\overline{g b g^{-1}}}\) and \(\symsf{\overline{b}}\), with \(\symsf{b = \tauu_1^{-1} \in \GB_1}\) and \(\symsf{g = \tauu_1^{-1} \in \GB_1}\).}
	\label{fig:conjugation}
\end{figure}

\begin{remark}
	The closures of conjugate braids are ambient isotopic as links. The braids themselves are not.
\end{remark}

\subsection{Markov move}

We observe that if \(b \in \GB_n\), then \(b\), \(b \tau_n \in \GB_{n+1}\) and \(b \tau_n^{-1} \in \GB_{n+1}\) have ambient isotopic closures. That is, we can add a strand and a crossing of that strand with another strand without changing the link type (of the closure). We can remove a strand and a crossing if that strand does not cross any other strand. It can be shown that adding the above mentioned strands anywhere between the existing strands is equivalent to adding the strands on the right.

\subsection{Markov's theorem}

\begin{thm}[Markov]
	Two braids whose closures are ambient isotopic to each other are related by a finite sequence of the following operations.
	\begin{enumerate}
		\item Braid equivalences.
		\item Conjugation.
		\item Markov moves.
	\end{enumerate}
\end{thm}
\begin{proof}
	Please refer to insert Birman citation.
\end{proof}



  	\printbibliography

\end{document}
